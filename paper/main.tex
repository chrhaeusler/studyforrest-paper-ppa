\documentclass[english,11pt]{article}
\usepackage[utf8]{inputenc}
\usepackage[T1]{fontenc}
\usepackage{stix2}
\usepackage{babel}
\usepackage[section]{placeins}
\usepackage{flafter}
\usepackage{comment}
\usepackage{breakurl}
\usepackage{svg}
\usepackage[labelfont=bf,font=small]{caption}
\usepackage{booktabs}
\usepackage{amsmath}
\usepackage{graphicx}
\usepackage{fancyhdr}
% \usepackage[disable]{todonotes}
\usepackage{todonotes}
\usepackage{units}
\usepackage[left=2.5cm,right=2.5cm,top=1cm,bottom=1cm,includeheadfoot]{geometry}
\usepackage[
colorlinks=true,
	citecolor=black,
  urlcolor=black,
	linkcolor=black
]{hyperref}
%\newcommand{\scidatalogo}{\includegraphics[height=36pt]{SciData_logo.jpg}}
\newcommand{\scidatalogo}{}
\pagestyle{fancy}
\fancyhf{}
\renewcommand{\headrulewidth}{0pt}
%\setlength{\headheight}{40pt}
%\lhead{\textsc{\scidatalogo}}
\usepackage[comma,super,sort&compress]{natbib}
\usepackage{xr}
\externaldocument{suppl}
% \usepackage{helvet}

% \renewcommand{\familydefault}{\sfdefault}

\begin{document}
\newcommand{\aoBodyAll}{66}
\newcommand{\aoBodyI}{6}
\newcommand{\aoBodyII}{12}
\newcommand{\aoBodyIII}{7}
\newcommand{\aoBodyIV}{12}
\newcommand{\aoBodyV}{2}
\newcommand{\aoBodyVI}{9}
\newcommand{\aoBodyVII}{13}
\newcommand{\aoBodyVIII}{5}

\newcommand{\aoBpartAll}{69}
\newcommand{\aoBpartI}{9}
\newcommand{\aoBpartII}{8}
\newcommand{\aoBpartIII}{6}
\newcommand{\aoBpartIV}{13}
\newcommand{\aoBpartV}{5}
\newcommand{\aoBpartVI}{7}
\newcommand{\aoBpartVII}{11}
\newcommand{\aoBpartVIII}{10}

\newcommand{\aoFaheadAll}{83}
\newcommand{\aoFaheadI}{12}
\newcommand{\aoFaheadII}{11}
\newcommand{\aoFaheadIII}{10}
\newcommand{\aoFaheadIV}{5}
\newcommand{\aoFaheadV}{9}
\newcommand{\aoFaheadVI}{13}
\newcommand{\aoFaheadVII}{12}
\newcommand{\aoFaheadVIII}{11}

\newcommand{\aoFgadlrdiffAll}{180k}
\newcommand{\aoFgadlrdiffI}{22k}
\newcommand{\aoFgadlrdiffII}{22k}
\newcommand{\aoFgadlrdiffIII}{21k}
\newcommand{\aoFgadlrdiffIV}{24k}
\newcommand{\aoFgadlrdiffV}{23k}
\newcommand{\aoFgadlrdiffVI}{21k}
\newcommand{\aoFgadlrdiffVII}{27k}
\newcommand{\aoFgadlrdiffVIII}{16k}

\newcommand{\aoFgadrmsAll}{180k}
\newcommand{\aoFgadrmsI}{22k}
\newcommand{\aoFgadrmsII}{22k}
\newcommand{\aoFgadrmsIII}{21k}
\newcommand{\aoFgadrmsIV}{24k}
\newcommand{\aoFgadrmsV}{23k}
\newcommand{\aoFgadrmsVI}{21k}
\newcommand{\aoFgadrmsVII}{27k}
\newcommand{\aoFgadrmsVIII}{16k}

\newcommand{\aoFurnAll}{50}
\newcommand{\aoFurnI}{8}
\newcommand{\aoFurnII}{5}
\newcommand{\aoFurnIII}{2}
\newcommand{\aoFurnIV}{5}
\newcommand{\aoFurnV}{7}
\newcommand{\aoFurnVI}{10}
\newcommand{\aoFurnVII}{7}
\newcommand{\aoFurnVIII}{6}

\newcommand{\aoGeoAll}{125}
\newcommand{\aoGeoI}{16}
\newcommand{\aoGeoII}{17}
\newcommand{\aoGeoIII}{11}
\newcommand{\aoGeoIV}{32}
\newcommand{\aoGeoV}{0}
\newcommand{\aoGeoVI}{15}
\newcommand{\aoGeoVII}{18}
\newcommand{\aoGeoVIII}{16}

\newcommand{\aoGroomAll}{105}
\newcommand{\aoGroomI}{12}
\newcommand{\aoGroomII}{11}
\newcommand{\aoGroomIII}{8}
\newcommand{\aoGroomIV}{5}
\newcommand{\aoGroomV}{8}
\newcommand{\aoGroomVI}{25}
\newcommand{\aoGroomVII}{28}
\newcommand{\aoGroomVIII}{8}

\newcommand{\aoObjAll}{284}
\newcommand{\aoObjI}{39}
\newcommand{\aoObjII}{34}
\newcommand{\aoObjIII}{27}
\newcommand{\aoObjIV}{44}
\newcommand{\aoObjV}{29}
\newcommand{\aoObjVI}{42}
\newcommand{\aoObjVII}{32}
\newcommand{\aoObjVIII}{37}

\newcommand{\aoSenewAll}{86}
\newcommand{\aoSenewI}{11}
\newcommand{\aoSenewII}{15}
\newcommand{\aoSenewIII}{12}
\newcommand{\aoSenewIV}{4}
\newcommand{\aoSenewV}{15}
\newcommand{\aoSenewVI}{10}
\newcommand{\aoSenewVII}{16}
\newcommand{\aoSenewVIII}{3}

\newcommand{\aoSeoldAll}{37}
\newcommand{\aoSeoldI}{2}
\newcommand{\aoSeoldII}{5}
\newcommand{\aoSeoldIII}{1}
\newcommand{\aoSeoldIV}{4}
\newcommand{\aoSeoldV}{2}
\newcommand{\aoSeoldVI}{9}
\newcommand{\aoSeoldVII}{8}
\newcommand{\aoSeoldVIII}{6}

\newcommand{\aoSexfAll}{108}
\newcommand{\aoSexfI}{14}
\newcommand{\aoSexfII}{22}
\newcommand{\aoSexfIII}{6}
\newcommand{\aoSexfIV}{6}
\newcommand{\aoSexfV}{13}
\newcommand{\aoSexfVI}{10}
\newcommand{\aoSexfVII}{23}
\newcommand{\aoSexfVIII}{14}

\newcommand{\aoSexmAll}{403}
\newcommand{\aoSexmI}{41}
\newcommand{\aoSexmII}{68}
\newcommand{\aoSexmIII}{38}
\newcommand{\aoSexmIV}{102}
\newcommand{\aoSexmV}{45}
\newcommand{\aoSexmVI}{42}
\newcommand{\aoSexmVII}{42}
\newcommand{\aoSexmVIII}{25}

\newcommand{\aoSexuAll}{17}
\newcommand{\aoSexuI}{0}
\newcommand{\aoSexuII}{3}
\newcommand{\aoSexuIII}{1}
\newcommand{\aoSexuIV}{2}
\newcommand{\aoSexuV}{3}
\newcommand{\aoSexuVI}{1}
\newcommand{\aoSexuVII}{5}
\newcommand{\aoSexuVIII}{2}

\newcommand{\aoVlochAll}{89}
\newcommand{\aoVlochI}{10}
\newcommand{\aoVlochII}{31}
\newcommand{\aoVlochIII}{2}
\newcommand{\aoVlochIV}{23}
\newcommand{\aoVlochV}{4}
\newcommand{\aoVlochVI}{18}
\newcommand{\aoVlochVII}{1}
\newcommand{\aoVnocutAll}{148}
\newcommand{\aoVnocutI}{30}
\newcommand{\aoVnocutII}{13}
\newcommand{\aoVnocutIII}{21}
\newcommand{\aoVnocutIV}{15}
\newcommand{\aoVnocutV}{27}
\newcommand{\aoVnocutVI}{9}
\newcommand{\aoVnocutVII}{17}
\newcommand{\aoVnocutVIII}{16}

\newcommand{\aoVpenewAll}{386}
\newcommand{\aoVpenewI}{31}
\newcommand{\aoVpenewII}{38}
\newcommand{\aoVpenewIII}{72}
\newcommand{\aoVpenewIV}{90}
\newcommand{\aoVpenewV}{89}
\newcommand{\aoVpenewVI}{33}
\newcommand{\aoVpenewVII}{24}
\newcommand{\aoVpenewVIII}{9}

\newcommand{\aoVpeoldAll}{208}
\newcommand{\aoVpeoldI}{25}
\newcommand{\aoVpeoldII}{61}
\newcommand{\aoVpeoldIII}{13}
\newcommand{\aoVpeoldIV}{1}
\newcommand{\aoVpeoldV}{32}
\newcommand{\aoVpeoldVI}{29}
\newcommand{\aoVpeoldVII}{47}
\newcommand{\aoVsenewAll}{96}
\newcommand{\aoVsenewI}{11}
\newcommand{\aoVsenewII}{14}
\newcommand{\aoVsenewIII}{17}
\newcommand{\aoVsenewIV}{4}
\newcommand{\aoVsenewV}{17}
\newcommand{\aoVsenewVI}{9}
\newcommand{\aoVsenewVII}{21}
\newcommand{\aoVsenewVIII}{3}

\newcommand{\aoVseoldAll}{90}
\newcommand{\aoVseoldI}{7}
\newcommand{\aoVseoldII}{11}
\newcommand{\aoVseoldIII}{3}
\newcommand{\aoVseoldIV}{7}
\newcommand{\aoVseoldV}{7}
\newcommand{\aoVseoldVI}{23}
\newcommand{\aoVseoldVII}{15}
\newcommand{\aoVseoldVIII}{17}


\newcommand{\avBodyAll}{66}
\newcommand{\avBodyI}{6}
\newcommand{\avBodyII}{12}
\newcommand{\avBodyIII}{7}
\newcommand{\avBodyIV}{12}
\newcommand{\avBodyV}{2}
\newcommand{\avBodyVI}{9}
\newcommand{\avBodyVII}{13}
\newcommand{\avBodyVIII}{5}

\newcommand{\avBpartAll}{69}
\newcommand{\avBpartI}{9}
\newcommand{\avBpartII}{8}
\newcommand{\avBpartIII}{6}
\newcommand{\avBpartIV}{13}
\newcommand{\avBpartV}{5}
\newcommand{\avBpartVI}{7}
\newcommand{\avBpartVII}{11}
\newcommand{\avBpartVIII}{10}

\newcommand{\avFaheadAll}{83}
\newcommand{\avFaheadI}{12}
\newcommand{\avFaheadII}{11}
\newcommand{\avFaheadIII}{10}
\newcommand{\avFaheadIV}{5}
\newcommand{\avFaheadV}{9}
\newcommand{\avFaheadVI}{13}
\newcommand{\avFaheadVII}{12}
\newcommand{\avFaheadVIII}{11}

\newcommand{\avFgavgerlrAll}{180k}
\newcommand{\avFgavgerlrI}{22k}
\newcommand{\avFgavgerlrII}{22k}
\newcommand{\avFgavgerlrIII}{22k}
\newcommand{\avFgavgerlrIV}{24k}
\newcommand{\avFgavgerlrV}{23k}
\newcommand{\avFgavgerlrVI}{22k}
\newcommand{\avFgavgerlrVII}{27k}
\newcommand{\avFgavgerlrVIII}{16k}

\newcommand{\avFgavgerlrdiffAll}{180k}
\newcommand{\avFgavgerlrdiffI}{22k}
\newcommand{\avFgavgerlrdiffII}{22k}
\newcommand{\avFgavgerlrdiffIII}{22k}
\newcommand{\avFgavgerlrdiffIV}{24k}
\newcommand{\avFgavgerlrdiffV}{23k}
\newcommand{\avFgavgerlrdiffVI}{22k}
\newcommand{\avFgavgerlrdiffVII}{27k}
\newcommand{\avFgavgerlrdiffVIII}{16k}

\newcommand{\avFgavgermlAll}{180k}
\newcommand{\avFgavgermlI}{22k}
\newcommand{\avFgavgermlII}{22k}
\newcommand{\avFgavgermlIII}{22k}
\newcommand{\avFgavgermlIV}{24k}
\newcommand{\avFgavgermlV}{23k}
\newcommand{\avFgavgermlVI}{22k}
\newcommand{\avFgavgermlVII}{27k}
\newcommand{\avFgavgermlVIII}{16k}

\newcommand{\avFgavgerpdAll}{180k}
\newcommand{\avFgavgerpdI}{22k}
\newcommand{\avFgavgerpdII}{22k}
\newcommand{\avFgavgerpdIII}{22k}
\newcommand{\avFgavgerpdIV}{24k}
\newcommand{\avFgavgerpdV}{23k}
\newcommand{\avFgavgerpdVI}{22k}
\newcommand{\avFgavgerpdVII}{27k}
\newcommand{\avFgavgerpdVIII}{16k}

\newcommand{\avFgavgerrmsAll}{180k}
\newcommand{\avFgavgerrmsI}{22k}
\newcommand{\avFgavgerrmsII}{22k}
\newcommand{\avFgavgerrmsIII}{22k}
\newcommand{\avFgavgerrmsIV}{24k}
\newcommand{\avFgavgerrmsV}{23k}
\newcommand{\avFgavgerrmsVI}{22k}
\newcommand{\avFgavgerrmsVII}{27k}
\newcommand{\avFgavgerrmsVIII}{16k}

\newcommand{\avFgavgerudAll}{180k}
\newcommand{\avFgavgerudI}{22k}
\newcommand{\avFgavgerudII}{22k}
\newcommand{\avFgavgerudIII}{22k}
\newcommand{\avFgavgerudIV}{24k}
\newcommand{\avFgavgerudV}{23k}
\newcommand{\avFgavgerudVI}{22k}
\newcommand{\avFgavgerudVII}{27k}
\newcommand{\avFgavgerudVIII}{16k}

\newcommand{\avFurnAll}{50}
\newcommand{\avFurnI}{8}
\newcommand{\avFurnII}{5}
\newcommand{\avFurnIII}{2}
\newcommand{\avFurnIV}{5}
\newcommand{\avFurnV}{7}
\newcommand{\avFurnVI}{10}
\newcommand{\avFurnVII}{7}
\newcommand{\avFurnVIII}{6}

\newcommand{\avGeoAll}{125}
\newcommand{\avGeoI}{16}
\newcommand{\avGeoII}{17}
\newcommand{\avGeoIII}{11}
\newcommand{\avGeoIV}{32}
\newcommand{\avGeoV}{0}
\newcommand{\avGeoVI}{15}
\newcommand{\avGeoVII}{18}
\newcommand{\avGeoVIII}{16}

\newcommand{\avGroomAll}{105}
\newcommand{\avGroomI}{12}
\newcommand{\avGroomII}{11}
\newcommand{\avGroomIII}{8}
\newcommand{\avGroomIV}{5}
\newcommand{\avGroomV}{8}
\newcommand{\avGroomVI}{25}
\newcommand{\avGroomVII}{28}
\newcommand{\avGroomVIII}{8}

\newcommand{\avObjAll}{284}
\newcommand{\avObjI}{39}
\newcommand{\avObjII}{34}
\newcommand{\avObjIII}{27}
\newcommand{\avObjIV}{44}
\newcommand{\avObjV}{29}
\newcommand{\avObjVI}{42}
\newcommand{\avObjVII}{32}
\newcommand{\avObjVIII}{37}

\newcommand{\avSenewAll}{86}
\newcommand{\avSenewI}{11}
\newcommand{\avSenewII}{15}
\newcommand{\avSenewIII}{12}
\newcommand{\avSenewIV}{4}
\newcommand{\avSenewV}{15}
\newcommand{\avSenewVI}{10}
\newcommand{\avSenewVII}{16}
\newcommand{\avSenewVIII}{3}

\newcommand{\avSeoldAll}{37}
\newcommand{\avSeoldI}{2}
\newcommand{\avSeoldII}{5}
\newcommand{\avSeoldIII}{1}
\newcommand{\avSeoldIV}{4}
\newcommand{\avSeoldV}{2}
\newcommand{\avSeoldVI}{9}
\newcommand{\avSeoldVII}{8}
\newcommand{\avSeoldVIII}{6}

\newcommand{\avSexfAll}{108}
\newcommand{\avSexfI}{14}
\newcommand{\avSexfII}{22}
\newcommand{\avSexfIII}{6}
\newcommand{\avSexfIV}{6}
\newcommand{\avSexfV}{13}
\newcommand{\avSexfVI}{10}
\newcommand{\avSexfVII}{23}
\newcommand{\avSexfVIII}{14}

\newcommand{\avSexmAll}{403}
\newcommand{\avSexmI}{41}
\newcommand{\avSexmII}{68}
\newcommand{\avSexmIII}{38}
\newcommand{\avSexmIV}{102}
\newcommand{\avSexmV}{45}
\newcommand{\avSexmVI}{42}
\newcommand{\avSexmVII}{42}
\newcommand{\avSexmVIII}{25}

\newcommand{\avSexuAll}{17}
\newcommand{\avSexuI}{0}
\newcommand{\avSexuII}{3}
\newcommand{\avSexuIII}{1}
\newcommand{\avSexuIV}{2}
\newcommand{\avSexuV}{3}
\newcommand{\avSexuVI}{1}
\newcommand{\avSexuVII}{5}
\newcommand{\avSexuVIII}{2}

\newcommand{\avVlochAll}{89}
\newcommand{\avVlochI}{10}
\newcommand{\avVlochII}{31}
\newcommand{\avVlochIII}{2}
\newcommand{\avVlochIV}{23}
\newcommand{\avVlochV}{4}
\newcommand{\avVlochVI}{18}
\newcommand{\avVlochVII}{1}
\newcommand{\avVnocutAll}{148}
\newcommand{\avVnocutI}{30}
\newcommand{\avVnocutII}{13}
\newcommand{\avVnocutIII}{21}
\newcommand{\avVnocutIV}{15}
\newcommand{\avVnocutV}{27}
\newcommand{\avVnocutVI}{9}
\newcommand{\avVnocutVII}{17}
\newcommand{\avVnocutVIII}{16}

\newcommand{\avVpenewAll}{386}
\newcommand{\avVpenewI}{31}
\newcommand{\avVpenewII}{38}
\newcommand{\avVpenewIII}{72}
\newcommand{\avVpenewIV}{90}
\newcommand{\avVpenewV}{89}
\newcommand{\avVpenewVI}{33}
\newcommand{\avVpenewVII}{24}
\newcommand{\avVpenewVIII}{9}

\newcommand{\avVpeoldAll}{208}
\newcommand{\avVpeoldI}{25}
\newcommand{\avVpeoldII}{61}
\newcommand{\avVpeoldIII}{13}
\newcommand{\avVpeoldIV}{1}
\newcommand{\avVpeoldV}{32}
\newcommand{\avVpeoldVI}{29}
\newcommand{\avVpeoldVII}{47}
\newcommand{\avVsenewAll}{96}
\newcommand{\avVsenewI}{11}
\newcommand{\avVsenewII}{14}
\newcommand{\avVsenewIII}{17}
\newcommand{\avVsenewIV}{4}
\newcommand{\avVsenewV}{17}
\newcommand{\avVsenewVI}{9}
\newcommand{\avVsenewVII}{21}
\newcommand{\avVsenewVIII}{3}

\newcommand{\avVseoldAll}{90}
\newcommand{\avVseoldI}{7}
\newcommand{\avVseoldII}{11}
\newcommand{\avVseoldIII}{3}
\newcommand{\avVseoldIV}{7}
\newcommand{\avVseoldV}{7}
\newcommand{\avVseoldVI}{23}
\newcommand{\avVseoldVII}{15}
\newcommand{\avVseoldVIII}{17}


\newcommand{\anAll}{17}
\newcommand{\anI}{2}
\newcommand{\anII}{2}
\newcommand{\anIII}{2}
\newcommand{\anIV}{0}
\newcommand{\anV}{3}
\newcommand{\anVI}{3}
\newcommand{\anVII}{4}
\newcommand{\anVIII}{1}

\newcommand{\anBodyAll}{66}
\newcommand{\anBodyI}{6}
\newcommand{\anBodyII}{12}
\newcommand{\anBodyIII}{7}
\newcommand{\anBodyIV}{12}
\newcommand{\anBodyV}{2}
\newcommand{\anBodyVI}{9}
\newcommand{\anBodyVII}{13}
\newcommand{\anBodyVIII}{5}

\newcommand{\anBodypartAll}{69}
\newcommand{\anBodypartI}{9}
\newcommand{\anBodypartII}{8}
\newcommand{\anBodypartIII}{6}
\newcommand{\anBodypartIV}{13}
\newcommand{\anBodypartV}{5}
\newcommand{\anBodypartVI}{7}
\newcommand{\anBodypartVII}{11}
\newcommand{\anBodypartVIII}{10}

\newcommand{\anFaceAll}{47}
\newcommand{\anFaceI}{7}
\newcommand{\anFaceII}{7}
\newcommand{\anFaceIII}{6}
\newcommand{\anFaceIV}{1}
\newcommand{\anFaceV}{7}
\newcommand{\anFaceVI}{9}
\newcommand{\anFaceVII}{6}
\newcommand{\anFaceVIII}{4}

\newcommand{\anFemaleAll}{31}
\newcommand{\anFemaleI}{12}
\newcommand{\anFemaleII}{8}
\newcommand{\anFemaleIII}{0}
\newcommand{\anFemaleIV}{3}
\newcommand{\anFemaleV}{0}
\newcommand{\anFemaleVI}{3}
\newcommand{\anFemaleVII}{2}
\newcommand{\anFemaleVIII}{3}

\newcommand{\anFemalesAll}{3}
\newcommand{\anFemalesI}{0}
\newcommand{\anFemalesII}{0}
\newcommand{\anFemalesIII}{0}
\newcommand{\anFemalesIV}{2}
\newcommand{\anFemalesV}{0}
\newcommand{\anFemalesVI}{0}
\newcommand{\anFemalesVII}{1}
\newcommand{\anFemalesVIII}{0}

\newcommand{\anFnameAll}{74}
\newcommand{\anFnameI}{2}
\newcommand{\anFnameII}{14}
\newcommand{\anFnameIII}{6}
\newcommand{\anFnameIV}{1}
\newcommand{\anFnameV}{13}
\newcommand{\anFnameVI}{7}
\newcommand{\anFnameVII}{20}
\newcommand{\anFnameVIII}{11}

\newcommand{\anFurnitureAll}{50}
\newcommand{\anFurnitureI}{8}
\newcommand{\anFurnitureII}{5}
\newcommand{\anFurnitureIII}{2}
\newcommand{\anFurnitureIV}{5}
\newcommand{\anFurnitureV}{7}
\newcommand{\anFurnitureVI}{10}
\newcommand{\anFurnitureVII}{7}
\newcommand{\anFurnitureVIII}{6}

\newcommand{\anGeoAll}{125}
\newcommand{\anGeoI}{16}
\newcommand{\anGeoII}{17}
\newcommand{\anGeoIII}{11}
\newcommand{\anGeoIV}{32}
\newcommand{\anGeoV}{0}
\newcommand{\anGeoVI}{15}
\newcommand{\anGeoVII}{18}
\newcommand{\anGeoVIII}{16}

\newcommand{\anGeoroomAll}{105}
\newcommand{\anGeoroomI}{12}
\newcommand{\anGeoroomII}{11}
\newcommand{\anGeoroomIII}{8}
\newcommand{\anGeoroomIV}{5}
\newcommand{\anGeoroomV}{8}
\newcommand{\anGeoroomVI}{25}
\newcommand{\anGeoroomVII}{28}
\newcommand{\anGeoroomVIII}{8}

\newcommand{\anHeadAll}{36}
\newcommand{\anHeadI}{5}
\newcommand{\anHeadII}{4}
\newcommand{\anHeadIII}{4}
\newcommand{\anHeadIV}{4}
\newcommand{\anHeadV}{2}
\newcommand{\anHeadVI}{4}
\newcommand{\anHeadVII}{6}
\newcommand{\anHeadVIII}{7}

\newcommand{\anMaleAll}{89}
\newcommand{\anMaleI}{15}
\newcommand{\anMaleII}{18}
\newcommand{\anMaleIII}{9}
\newcommand{\anMaleIV}{18}
\newcommand{\anMaleV}{7}
\newcommand{\anMaleVI}{8}
\newcommand{\anMaleVII}{9}
\newcommand{\anMaleVIII}{5}

\newcommand{\anMalesAll}{23}
\newcommand{\anMalesI}{2}
\newcommand{\anMalesII}{11}
\newcommand{\anMalesIII}{4}
\newcommand{\anMalesIV}{3}
\newcommand{\anMalesV}{2}
\newcommand{\anMalesVI}{0}
\newcommand{\anMalesVII}{1}
\newcommand{\anMalesVIII}{0}

\newcommand{\anMnameAll}{291}
\newcommand{\anMnameI}{24}
\newcommand{\anMnameII}{39}
\newcommand{\anMnameIII}{25}
\newcommand{\anMnameIV}{81}
\newcommand{\anMnameV}{36}
\newcommand{\anMnameVI}{34}
\newcommand{\anMnameVII}{32}
\newcommand{\anMnameVIII}{20}

\newcommand{\anObjectAll}{232}
\newcommand{\anObjectI}{36}
\newcommand{\anObjectII}{22}
\newcommand{\anObjectIII}{20}
\newcommand{\anObjectIV}{30}
\newcommand{\anObjectV}{25}
\newcommand{\anObjectVI}{37}
\newcommand{\anObjectVII}{26}
\newcommand{\anObjectVIII}{36}

\newcommand{\anObjectsAll}{52}
\newcommand{\anObjectsI}{3}
\newcommand{\anObjectsII}{12}
\newcommand{\anObjectsIII}{7}
\newcommand{\anObjectsIV}{14}
\newcommand{\anObjectsV}{4}
\newcommand{\anObjectsVI}{5}
\newcommand{\anObjectsVII}{6}
\newcommand{\anObjectsVIII}{1}

\newcommand{\anPersonsAll}{17}
\newcommand{\anPersonsI}{0}
\newcommand{\anPersonsII}{3}
\newcommand{\anPersonsIII}{1}
\newcommand{\anPersonsIV}{2}
\newcommand{\anPersonsV}{3}
\newcommand{\anPersonsVI}{1}
\newcommand{\anPersonsVII}{5}
\newcommand{\anPersonsVIII}{2}

\newcommand{\anSettingnewAll}{86}
\newcommand{\anSettingnewI}{11}
\newcommand{\anSettingnewII}{15}
\newcommand{\anSettingnewIII}{12}
\newcommand{\anSettingnewIV}{4}
\newcommand{\anSettingnewV}{15}
\newcommand{\anSettingnewVI}{10}
\newcommand{\anSettingnewVII}{16}
\newcommand{\anSettingnewVIII}{3}

\newcommand{\anSettingrecAll}{37}
\newcommand{\anSettingrecI}{2}
\newcommand{\anSettingrecII}{5}
\newcommand{\anSettingrecIII}{1}
\newcommand{\anSettingrecIV}{4}
\newcommand{\anSettingrecV}{2}
\newcommand{\anSettingrecVI}{9}
\newcommand{\anSettingrecVII}{8}
\newcommand{\anSettingrecVIII}{6}



% title has to be < 110 chars incl. spaces at the moment 106 Chars
\title{Processing of visual and non-visual naturalistic spatial information in
  the "parahippocampal place area"}

\author{
    Christian~O.~Häusler\textsuperscript{1,2{*}},
    Simon B. Eickhoff\textsuperscript{1,2},
    Michael Hanke\textsuperscript{1,2}}
% https://www.nature.com/sdata/publish/for-authors#other-formats

\maketitle
\thispagestyle{fancy}

\noindent
1. Psychoinformatics Lab, Institute of Neuroscience and Medicine, Brain \&
Behaviour (INM-7), Research Centre Jülich, Jülich, Germany\\
\noindent
2. Institute of Systems Neuroscience, Medical Faculty, Heinrich Heine University,
Düsseldorf, Germany\\
{*}corresponding author: Christian Olaf Häusler (der.haeusler@gmx.net)


\begin{abstract}
% < 170 words new analysis of existing data of interest to a broad section of
% our audience highlighting innovative examples of data reuse may be used to
% present compelling new findings & conclusions derived from published.
The ``parahippocampal place area'' (PPA) in the human ventral visual stream
exhibits increased hemodynamic activity correlated with the perception of
landscape photos compared to faces or objects.
% AV, AD: operationalization
Here, we investigate the perception of scene-related, spatial information embedded in two naturalistic stimuli.
%
The same 14 participants were watching a Hollywood movie and listening to its
audio-description as part of the open-data resource \textit{studyforrest.org}.
% auditory localizer?
%in order to explore if a naturalistic auditory narrative could be used to
%localize a ``visual area''.
We model hemodynamic activity based on annotations of selected stimulus
features,
% VIS
and compare results to a block-design visual localizer.
% no refs in abstract (probably)
%\citep{sengupta2016extension}.  results: group AV
On a group level, increased activation correlating with visual spatial information
occurring in the movie is overlapping with a traditionally localized PPA.
% results: group AD
Activation correlating with semantic spatial information occurring in the
audio-description is more restricted to the anterior PPA.
% results: individual AD
On an individual level, we find significant bilateral activity in the PPA
of nine individuals and unilateral activity in one individual.
% conclusion: generalizability
Results suggest that activation in the PPA generalizes to spatial information
embedded in a movie and an auditory narrative, and may
% conclusion: PPA has subregions
call for considering a functional subdivision of the PPA.  \end{abstract}

%\section*{Keywords}
% maximal 8
% fMRI, naturalistic stimulus, spatial perception, vision, language, speech, narrative, studyforrest

\pagebreak[4]


\section*{Introduction}

% Limit for main text (intro, results and discussion) = ca. 3k words

% brain mapping via fMRI
Studies in the field of neuropsychology and neuroimaging
(\citep[e.g.,][]{penfield1950cerebral, fox1984noninvasive}) have shown that
different parts of the brain are specialized for different perceptual and
cognitive functions.
% occipital cortex for vision -> two pathways
The occipital cortex is considered to be primarily involved in the early stages
of visual perception and giving rise to two distinct, but interacting, pathways
that serve different functions:
% dorsal vs ventral pathway
a) a dorsal stream (the ``how pathway'') that leads into the parietal lobe and
supports visual guidance of action, and b) a ventral stream (the ``what
pathway'') that leads into the temporal lobe and supports conscious perception
and recognition \citep{goodale1992separate, milner2017two, ungerleider1982two}.
% PPA
A classic example of a higher-level visual area in the ventral pathway is the
``parahippocampal place area'' (PPA) \citep{epstein1998ppa,
epstein1999parahippocampal}.
% anatomical location != functional location
The PPA is located in the posterior parahippocampal gyrus including adjacent
regions of the fusiform gyrus and anterior lingual gyrus
\citep{epstein2008parahippocampal}.
% neural correlate of scene perception
Increased hemodynamic activity is observed in the PPA when participants view
pictures of landscapes, buildings or landmarks, compared to e.g. pictures of
faces or tools, during blood oxygenation level-dependent functional magnetic
resonance imaging (BOLD fMRI) (\citep[see reviews][]{epstein2014neural,
aminoff2013role}).
% \citep{aguirre1998area, epstein2014neural, epstein1998ppa, troiani2012object}.

% literature review overview: imagination & haptic exploration; van der Hurk: 64
% sounds x 1.8 seconds; 20 sighted, 14 congenitally blind
Increased hemodynamic activity in the PPA generalizes from pictures to mental
imagery of landscapes \citep{ocraven2000mental}, haptic exploration of
scenes constructed from LEGO blocks \citep{wolbers2011modality}, and
scene-related sounds \citep{van2017development}.
% % O'Craven: watching pictures
In a study conducted by \citet{ocraven2000mental} participants viewed
alternating blocks of pictures showing familiar places and famous faces during
an initial experimental paradigm.
% O'Craven: mental imagery
In a subsequent paradigm, participants were instructed to ``form a vivid mental
image'' of the previously viewed pictures.
% O'Craven results
The PPA showed increased activation during imagination of places compared to
faces but the imagination task showed a smaller activation level compared to the
perceptual task.
% Wolbers: haptic exploration
In a block design study conducted by \citet{wolbers2011modality} the PPA of
sighted as well as blind participants showed increased activation during a
delayed match-to-sample task of haptically explored scenes constructed from LEGO
bricks compared to abstract geometric objects.
% Wolbers: connectivity analysis

% Aziz (2008): place related sentences
To our knowledge only one study \citep{aziz2008modulation} compared hemodynamic
activity levels in the PPA that were correlated with different semantic
categories occurring in \textit{speech}.
% Aziz' stimuli; ``The Taj Mahal faces a long thin reflecting pool'', ``Marilyn
% Monroe has a large square jaw'', ``The television has a long antenna''
\citet{aziz2008modulation} used sentences that described famous or generic
places, faces, or objects.
% Aziz' tasks:
Participants were instructed to press a button whenever the sentence described
an inaccurate or improbable fact (e.g. ``Marilyn Monroe has a large square
jaw'').
% Aziz' results:
Activation in the left, but not right, PPA was significantly reduced when
participants listened to place-related sentences compared to listening to
face-related sentences. Moreover, this effect was only observed in sentences
involving famous places.

% summary if literature review
Taken together, the literature suggests that the PPA does not exclusively respond
to visually presented scene-related, spatial information.
% three things in common
However, all reviewed studies share three common aspects:
% designed stimulus set
1) they employed a small set of carefully chosen and conceptualized stimuli,
% block-design
2) they exclusively used a block-design paradigm, and
% task
3) they employed an explicit (perceptual) judgement task.
% conceptualized stimuli and block: disadvantages
Block-design studies that use conceptualized stimuli and a task have the
advantage of controlling confounding variables (e.g. color, luminance, size,
spatial frequencies, sentence length), maximizing detection power, and keeping
participants paying attention to the stimuli.
% conceptualized stimuli and block: disadvantages
Nevertheless, small sets of conceptualized stimuli and block-design paradigms
lack external and ecological validity \citep{westfall2016fixing,
hasson2004intersubject} because they poorly resemble how we, free of an explicit
perceptual task, experience our rich, multidimensional and continuous
environment that our brains are accommodated to
\citep{sonkusare2019naturalistic}.

% open question
In this study, we investigate whether increased hemodynamic activity in the PPA
that is usually detected by contrasting blocks of pictures is also present under
more natural conditions.
% we use 2 naturalistic stimuli
To answer this question, we operationalized the perception of both
\textit{visual and auditory} spatial information using two naturalistic stimuli
(\citep[see reviews][]{hamilton2018revolution, hasson2008neurocinematics,
sonkusare2019naturalistic}).
% AV stimulus
The current operationalization of visual spatial perception is based on an
annotation of cuts and depicted locations in the audio-visual movie ``Forrest
Gump'' \citep{haeusler2016cutanno}, while
% AD stimulus
the operationalization of non-visual spatial perception is based on an
annotation of speech occurring in the movie's audio-description
\citep{haeusler2021studyforrest}.
% pictures -> movie -> audio-description
The movie stimulus shares the stimulation in the visual domain with classical
localizer stimuli, while featuring real-life-like visual complexity and
naturalistic auditory stimulation. The audio-description maintains the
naturalistic nature of the movie stimulus, but limited to the auditory domain.
% current GLM contrasts
We applied model-based, mass-univariate analyses to BOLD fMRI data from both
naturalistic stimuli \citep{hanke2016simultaneous, hanke2014audiomovie},
available from the open-data resource
\href{http://www.studyforrest.org}{studyforrest.org} \citep{hanke2016aligned}.
% comparison to previous GLM
We compare current results to results of a previously performed model-based,
mass-univariate analysis that was applied to data from a conventional functional
localizer performed with the same set of participants
\citep{sengupta2016extension, sengupta2016extensiondata}.
% intro to hypo a
Similarly to the functional localizer, we currently also capitalize on events
that ought to evoke the cognitive processing of spatial information.
% hypo a
Thus, we hypothesized that our whole-brain analyses would reveal increased
hemodynamic activity in medial temporal regions that were functionally
identified as the PPA by the analysis of the localizer data.
%hypo b: individuals
We hypothesized further that a purely auditory stimulus could, in principle,
localize the PPA as an example of a ``visual area'' in individual persons,
% application in individuals
and may offer an alternative paradigm to assess brain functions in visually
impaired individuals.

% DISABLE RESULTS
%% results: group
%On a group average level,
%% operationalization in naturalistic stimuli
%results demonstrate that our approach to operationalize spatial perception
%during two naturalistic stimuli lead to increased hemodynamic activity in the
%PPA.
%% anterior vs. posterior
%On both a group and individual participant level, the comparison of results from the
%three datasets also suggests functional differences between the anterior and
%posterior PPA.
%% what does it mean
%The current data add evidence to the hypothesis that the PPA might consist of
%submodules that work together during visual perception of landscapes (cf.
%\citep{baldassano2013differential}).
%%
%Lastly, our results suggest that the PPA's anterior part can be localized in
%individual participants by contrasting high-level semantic information occurring in
%an auditory naturalistic stimulus.


\section*{Results}

% study goal
We investigated if spatial information embedded in two naturalistic stimuli
correlates with increased hemodynamic activity in the PPA.
%
Based on an annotation of cuts in the movie and an annotation of speech spoken
by the audio-description's narrator, we selected events in both stimuli that
should correlate with the perception of spatial information and contrasted them
with events that should correlate with non-spatial
perception to a lesser degree, or not at all (see Table~\ref{tab:events}).
% from supplementals
In order to test the robustness of our approach, we created multiple linear
model (GLM) $t$-contrasts for the movie and audio-description (see
Table~\ref{tab:contrasts}).
% subjectively assessed balance
For each stimulus, we chose a primary contrast for result presentation
based on a subjectively assessed balance of how
well the averaged events within categories represent spatial and non-spatial
information, and the number of events in the stimulus.
% group level
On a group average level, we report results from whole-brain analyses of
the movie's and the audio-description's primary contrasts and compare these
results to a traditional visual localizer \citep{sengupta2016extension}.
% individual level
For each individual, we also compare the activation pattern from naturalistic
auditory stimulation to the individual PPA localization with a block-design
stimulus.
% reference to supplementals
An evaluation of the results' robustness across all created contrasts on a group
average level is provided in the Supplementary Information (see
Figure~\ref{fig:stability-slices-volsurf}).
% NeuroVault
Unthresholded $Z$-maps of all contrasts on a group average level as well as
individual results of the primary $t$-contrasts co-registered to the
group-template (MNI152 space) can be found at
\href{https://neurovault.org/collections/KADGMGVZ/}{\url{neurovault.org/collections/KADGMGVZ}}.


\subsection*{Group analyses}

% av intro
First, we analyzed data from the movie that offered ecologically more valid
visual stimulation than a paradigm using blocks of pictures.
% AV primary contrast
The movie's primary $t$-contrast that compared cuts to a setting that was not
depicted before to cuts within a recurring setting (\texttt{vse\_new >
vpe\_old}) yielded three significant clusters (see
Table~\ref{tab:res-av-group1}).
% volume plots
Results are depicted as slices of brain volumes in
Figure~\ref{fig:grp_ao_av_c1_volsurf} a).
% surface plots
The surface plots depicted in Figure~\ref{fig:grp_ao_av_c1_volsurf} b) were
created by reconstructing the cortical surface of the MNI152 template using
\href{https://surfer.nmr.mgh.harvard.edu}{FreeSurfer} v7.1.1
\citep{dale1999cortical}, and projecting the $Z$-maps onto the surface using
FreeSurfer's 'mri\_vol2surf' command.
% PPA (others the clusters extend into).
One cluster spans across the midline and comprises parts of the intracalcarine
and cuneal cortex, the lingual gyrus and retrosplenial cortex, the occipital
and temporal fusiform gyrus, and the parahippocampal cortex (reported from
posterior to anterior) in both hemispheres.
% LOC
Two additional bilateral clusters are located in the superior lateral
occipital cortex.

% AD intro
Second, we analyzed data from the audio-description offering an exclusively
auditory stimulation.
% AD primary contrast
The primary $t$-contrast for the audio-description (\texttt{geo, groom} >
non-spatial noun categories) yielded six significant clusters (see
Table~\ref{tab:res-ao-group1}, Figure~\ref{fig:grp_ao_av_c1_volsurf}).
% overlap with ROI
Two bilateral clusters are located in the anterior part of the PPA group
overlap reported by \citet{sengupta2016extension}.
% in detail
Specifically, these clusters are located at the borders of the posterior
parahippocampal cortex, the occipital and temporal fusiform gyrus and lingual
gyrus.
% precuneus
Two additional bilateral clusters are apparent in the ventral precuneus
extending into the retrosplenial cortex.
% LOC
Finally, two bilateral clusters are located in the superior lateral occipital
cortex.

% figure group results, volume, primary AD & AV contrasts
\begin{comment}

\begin{figure*}[tbp] \centering
    \includegraphics[width=\linewidth]{figures/group-slices}

    \caption{Mixed-effects group-level (N=14) clusters ($Z$>3.4; $p$<.05,
      cluster-corrected) of activity correlated with the processing of spatial
      information are displayed on top of the MNI152 T1-weighted head template,
      with the acquisition field-of-view for the audio-description study
      highlighted.
      %
      The results of the audio-description's primary $t$-contrast (blue) that
      compares geometry-related nouns spoken by the narrator to non-spatial
      nouns (\texttt{geo, groom} > all non-spatial categories) are overlaid
      on the movie's primary $t$-contrast (red) that compares cuts to a
      setting depicted for the first time with cuts within a recurring setting
      (\texttt{vse\_new > vpe\_old}).
      %
      For comparison, the union of the individual PPA localizations reported
      by \citet{sengupta2016extension} are indicated as a black outline.
    }
    \label{fig:group-slices}
\end{figure*}

\end{comment}


% figure group results, volume & surface, primary AD & AV contrasts
\begin{figure*}[tbp]
\centering
    \includegraphics[width=\linewidth]{figures/group-slices-volsurf}

    \caption{Mixed-effects group-level (N=14) clusters ($Z$>3.4; $p$<.05,
      cluster-corrected) of activity correlated with the processing of spatial
      information.
      The results of the audio-description's primary $t$-contrast (blue) that
      compares geometry-related nouns spoken by the narrator to non-spatial
      nouns (\texttt{geo, groom} > all non-spatial categories) are overlaid
      on the movie's primary $t$-contrast (red) that compares cuts to a
      setting depicted for the first time to cuts within a recurring setting
      (\texttt{vse\_new > vpe\_old}).
      % volume
      a) results as brain slices on top of the MNI152 T1-weighted head
      template,
      with the acquisition field-of-view for the audio-description study
      highlighted.
      % PPA
      For comparison depicted as a black outline, the union of the
      individual PPA localizations reported by \citet{sengupta2016extension}
      that was spatially smoothed by applying a Gaussian kernel with full
      width at half maximum (FWHM) of \unit[2.0]{mm}.
      % surface
      b) results projected onto the reconstructed
      surface of the MNI152 T1-weighted brain template.
      After projection, the union of individual PPA localizations was
      spatially smoothed by a Gaussian kernel with FWHM of \unit[2.0]{mm}.
    }
    \label{fig:grp_ao_av_c1_volsurf}
\end{figure*}


\begin{table*}[tbp]
\caption{Clusters ($Z$-threshold $Z$>3.4; $p$<.05, cluster-corrected)
    of the primary $t$-contrast for the audio-visual movie comparing cuts to a
    setting depicted for the first time with cuts within a recurring setting
    (\texttt{vse\_new > vpe\_old}), sorted by size.
    The first brain structure given contains the voxel with the maximum $Z$-value,
    followed by brain structures from posterior to anterior, and partially
    covered areas (l.: left; r: right; c.: cortex; g.: gyrus; CoG: Center of
  Gravity).}
\label{tab:res-av-group1}
\resizebox{\linewidth}{!}{
\small
\begin{tabular}{rrrrrrrrrp{4.7cm}}
\toprule
& & & \multicolumn{3}{r}{Max (MNI)} & \multicolumn{3}{r}{CoG (MNI)} &
\\ \cmidrule{4-6} \cmidrule{7-9}
Voxels & $p_{corr.}$ & $Z_{max}$ & X & Y & Z  & X & Y & Z & Structure \\
\midrule
3003 & <.00001 & 5.31 & 22.5 & -45.5 & -12 & 4.53 & -63.3 & -3.7 & r.~lingual g.; r.~cuneal c., intracalcarine c., bilaterally occipital fusiform g., precuneus,temporal fusiform c., posterior parahippocampal c.  \\
154 & <.00001 & 4.46 & -35 & -83 & 28 & -32.8 & -86.2 & 21.4 & l.~superior
lateral occipital c. \\ % 0.000000656
121 & <.00001 & 4.65 & 25 & -80.5 & 25.5 & 23.7 & -83.8 & 25.4 & r.~superior
lateral occipital cortex \\ % 0.00000769
\bottomrule
\end{tabular}
}%resizebox
\end{table*}


% table group results primary AD contrast
\begin{table*}[tbp]
\caption{Clusters ($Z$-threshold $Z$>3.4; $p$<.05, cluster-corrected)
    of the primary $t$-contrast for the audio-description comparing
    geometry-related nouns to non-spatial nouns spoken by the
    audio-description's narrator (\texttt{geo, groom > all non-geo}), sorted by size.
    The first brain structure given contains the voxel with the maximum $Z$-value,
    followed by brain structures from posterior to anterior, and partially
    covered areas (l.: left; r: right; c.: cortex; g.: gyrus; CoG: Center of
    Gravity).}
    \label{tab:res-ao-group1}
\resizebox{\linewidth}{!}{
\small
\begin{tabular}{rrrrrrrrrp{4.7cm}}
\toprule
& & & \multicolumn{3}{r}{Max (MNI)} & \multicolumn{3}{r}{CoG (MNI)} &
\\ \cmidrule{4-6} \cmidrule{7-9}
Voxels & $p_{corr.}$ & $Z_{max}$ & X & Y & Z  & X & Y & Z & Structure \\
\midrule
188 & <.00001 & 4.48 & -17.5 & -65.5 & 25.5 & -14.7 & -59.1 & 15.2 & l.~precuneus \\ % 0.0000000596
164 & <.00001 & 4.47 & 17.5 & -58 & 23 & 15.6 & -55.6 & 16 & r.~precuneus;
\\ % 0.000000238
83 & .00013 & 4.48 & 27.5 & -43 & -17 & 27.2 & -41.1 & -14 & r.~occipito-temporal fusiform c.; posterior parahippocampal g. \\ % 0.000128
73 & .00031 & 3.93 & -22.5 & -43 & -12 & -23.9 & -43.6 & -11.2 & l.~lingual
g.; occipito-temporal fusiform g., posterior parahippocampal c. \\ % 0.000128
63 & .00082 & 4.1 & 40 & -75.5 & 30.5 & 40.9 & -76.3 & 28.6 & r.~superior
lateral occipital c. \\ % 0.000824
37 & .0129 & 4.24 & -37.5 & -78 & 33 & -38.4 & -79.5 & 28.9 & l.~superior
lateral occipital c. \\ % 0.0129
\bottomrule
\end{tabular}
}%resizebox
\end{table*}


\subsection*{Individual analyses}

% results in detail voxels of significant clusters in PPA ROI (and RSC
% liberally)
% sub-01 (m): PPA (l/r), AD (l/r), AV (-/r); RSC: AD (l/r), AV (-/-)
% sub-02 (m): PPA (l/r), AD (-/-), AV (l/-); RSC: AD (-/-), AV (-/-)
% sub-03 (f): PPA (l/r), AD (-/-), AV (-/r); RSC: AD (-/-), AV (-/-)
% sub-04 (f): PPA (-/r), AD (l/r), AV (-/r); RSC: AD (l/r), AV (-/-) !
% sub-05 (m): PPA (l/r), AD (-/-), AV (-/-); RSC: AD (-/-), AV (-/-)
% sub-06 (m): PPA (l/r), AD (l/r), AV (-/-); RSC: AD (l/-), AV (-/-)
% sub-09 (m): PPA (l/r), AD (l/-), AV (l/r); RSC: AD (l/r), AV (l/r)
% sub-14 (f): PPA (l/r), AD (l/r), AV (-/r); RSC: AD (l/r), AV (-/-)
% sub-15 (m): PPA (l/r), AD (l/r), AV (-/r); RSC: AD (l/r), AV (l/r)
% sub-16 (m): PPA (l/r), AD (l/r), AV (l/r); RSC: AD (l/r), AV (l/-)
% sub-17 (m): PPA (l/r), AD (l/r), AV (l/r); RSC: AD (l/r), AV (l/r)
% sub-18 (m): PPA (l/r), AD (l/r), AV (l/r); RSC: AD (l/r), AV (-/r)
% sub-19 (f): PPA (l/r), AD (l/r), AV (-/r); RSC: AD (l/r), AV (l/r)
% sub-20 (f): PPA (-/r), AD (-/-), AV (l/r); RSC: AD (-/-), AV (-/-)

% intro
Third, we inspected results from both naturalistic paradigms on the level of
individual participants, and compared them to the individual PPA localizations
provided by \citet{sengupta2016extension}, who reported
% Sengupta et al., 2016; unilateral in sub-04 and sub-20
bilateral parahippocampal clusters in 12 of 14 participants and unilateral right
clusters in two participants (\citep[see Table~3 in][]{sengupta2016extension}).
% one contrast to localize them all; "...and $Z$>3.4 for all"
Unlike \citet{sengupta2016extension}, who determined PPA clusters using three
candidate contrasts and a variable threshold, we used a single contrast for each
naturalistic stimulus and a uniform threshold for all participants.
% ref to figure
Figure~\ref{fig:subs-thresh-ppa} depicts thresholded $Z$-maps of the primary
movie and audio-description $t$-contrasts, in comparison to the results of a
conventional block-design localizer (see Figure~\ref{fig:subjs_ao_av_c1_surf}
for surface plost; unthresholded $Z$-maps are provided at
\href{https://neurovault.org/collections/KADGMGVZ/}{\url{neurovault.org/collections/KADGMGVZ}}).
% AV
Results of the primary movie contrast yielded bilateral clusters in five
participants, a unilateral right cluster in six participants (of which one
participant yielded a unilateral cluster in the visual localizer), and a
unilateral left cluster in one participant.
% difference to dedicated localizer
We find bilateral clusters for participant \texttt{sub-20}, whereas the
block-design localizer yielded only one cluster in the right hemisphere.
% report here in group
Results of the primary audio-description contrast yielded bilateral clusters in
nine participants that are within or overlapping with the block-design localizer
results.
% subj-04
In participant \texttt{sub-04}, two bilateral clusters are apparent, whereas
block-design localizer, and movie stimulus yielded only one cluster in the right
hemisphere.
% sub-09
For another participant (\texttt{sub-09}) the analysis yielded one cluster in
the left-hemispheric PPA.


\begin{figure*}[tbp]
\centering
    \includegraphics[width=\linewidth]{figures/subs-thresh-ppa}
    \caption{Fixed-effects individual-level GLM results ($Z$>3.4; $p$<.05,
        cluster-corrected).
        Individual brains are aligned via non-linear
        transformation to a study-specific T2* group template that is
        co-registered to the MNI152 template with an affine transformation (12
        degrees of freedom).
        The results of the audio-description's primary
        $t$-contrast (blue) that compares geometry related nouns to non-geometry related nouns spoken by the narrator
        (\texttt{geo, groom > all non-geo}) are overlaid over the movie's
        primary $t$-contrast (red) that compares cuts to a setting depicted for
        the first time with cuts within a recurring setting
        (\texttt{vse\_new > vpe\_old}).
        Black:
        outline of participant-specific PPA(s) reported by
        \citet{sengupta2016extension}.
        Light gray: The
        audio-description's field of view \citep{hanke2014audiomovie}.
        To facilitate comparisons across participants, we chose the same horizontal
        slice (x=-11) for all participants as this slice depicts voxels of
        significant clusters in almost all participants.
        The figure does not show voxels of the left cluster of the movie stimulus
        in sub-09 and sub-18, and voxels of the right cluster of the movie
        stimulus in sub-15.}
    \label{fig:subs-thresh-ppa}
\end{figure*}


% Bland-Altman-Plots
To illustrate the similarity of correlates of spatial processing in
naturalistic (audio-description) and conventional stimulation, independent of a
particular cluster-forming threshold, Bland-Altman plots for all participants
are shown in Figure~\ref{fig:bland-altman}.
% what it depicts
Each subplot visualizes the mean value and difference (localizer minus
audio-description) of all voxels in temporal and occipital cortices.
% maringal distributions
The marginal distributions of the mean scores indicate a general agreement of
both contrast scores across voxels in the PPA localization overlap as a
probabilistic indicator (blue), and the individual block-design localizer
results (red).
% lower right quadrant
Notably, 11 participants exhibit a pattern of increased $Z$-scores for the
naturalistic stimulus (lower right quadrant) that includes voxels labeled by the
block-design localizer, but also additional voxels.

\begin{figure*}[tbp]
\centering
    \includegraphics[scale=0.27]{figures/subjs_bland-altman.png}
    \caption{Bland-Altman-Plots for individual participants.
    The x-axes show the means of two spatially corresponding voxels in the
    unthresholded $Z$-map of the audio-description's primary contrast and
    unthresholded $Z$-map of the visual localizer (KDE plot on the top).
    The y-axes show the difference of two voxels (localizer minus
    audio-description; KDE plot on the right).
    The overlays depict voxels spatially constrained to the
    temporal and occipital cortex (gray; based on probabilistic Jülich
    Histological Atlas \citep{eickhoff2005toolbox, eickhoff2007assignment}),
    PPA overlap of all participants (blue),
    and individual PPA(s) (red).}
    \label{fig:bland-altman}
\end{figure*}


\section*{Discussion}

% Discussion: explains how results fill gap identified in intro

% typically done by recapitulating the results, discussing limitations, and then
% revealing how the central contribution may catalyze future
% progress

% provides caveats to the interpretation describes how the paper
% advances the field by providing new opportunities

%#1 previous studies
Several studies have reported increased hemodynamic activity in the PPA
attributed to the processing of scene-related, spatial information, for example,
when participants were watching static pictures of landscapes compared to
pictures of faces or objects \citep{epstein1998ppa, epstein1999parahippocampal}.
% auditory semantics unclear
However, reports regarding the correlates of processing spatial information in
verbal stimulation are less clear \citep{aziz2008modulation}.
% current study: similar
In line with previous studies, we investigated the hemodynamic response to
spatial information using a model-based, mass-univariate approach.
% current study: dissimilar
However, instead of using a conventional set of stimuli, assembled to
specifically and predominantly evoke the processing of spatial information, we
employed naturalistic stimuli, a movie and its audio-description, that were
designed for entertainment.

%#3 hypo
We hypothesized that due to the complex nature of the stimuli, unrelated factors
would be balanced across a large number of events, and make the bias of spatial
information accessible to a conventional model-based statistical analysis of BOLD
fMRI data.
% we need a detailed description -> annotation
This model-driven approach required a detailed annotation of the occurrence of
relevant stimulus features.
% event counts ftw
The annotation of both stimuli revealed the respective number of incidentally
occurring events to be similar to those of a conventional experimental paradigm
used to localize functional regions of interest.

%#4
We modeled hemodynamic responses correlating with spatial information embedded
in the naturalistic stimuli, capitalizing on conceptually similar, but
perceptually different stimulation events.
% group AV
On a group-average level, results for the movie show significantly
increased hemodynamic activity spatially overlapping with a conventionally
localized PPA but also extending into earlier visual cortices.
% group AD
Likewise, results for the audio-description identify significant activation in
the PPA but restricted to its anterior part.
% individual level
Bilateral clusters in 9 of 14
participants (of which \texttt{sub-04} shows only a right-lateralized PPA
in the block-design localizer results), and a unilateral
significant cluster in one participant, indicate that the group average results
are representative for the majority of individual participants.
% meaning: generalization to naturalistic stimuli
These findings suggest that increased activation in the PPA during the
perception of static pictures generalizes to the perception of spatial
information embedded in a movie or a purely auditory narrative.
%
Current results may partially deviate from \citet{sengupta2016extension}, due to
the uniform cluster forming threshold employed in this study versus their
adaptive procedure (bilateral clusters in 12 of 14 participants and a unilateral
right cluster in two participants (\texttt{sub-04}, \texttt{sub-20}).

%#11 question
The fact that clusters of responses to the auditory stimulus are spatially
restricted to the anterior part of clusters from both visual paradigms raises
the question if the revealed correlation patterns can be attributed to different
features inherent in the visual stimuli compared to a purely auditory stimulus.
% preliminary 'cause cross-scanner
Due to the nature of the datasets investigated here, such an attribution can
only be preliminary, because the auditory stimulation dataset also differs in
key acquisition properties (field-strength, resolution) from the comparison
datasets, representing a confound of undetermined impact.
% visual scenes: probably submodules
However, previous studies in the field of visual perception provide evidence
that the PPA can be divided into functionally subregions that might process
different stimulus features.
% posterior: functionality
The posterior PPA (pPPA) is functionally more responsive than the anterior PPA
(aPPA) to low-level features of scenes or (abstract) objects
\citep{baldassano2013differential, lescroart2019human, nasr2014thinking,
rajimehr2011parahippocampal}.
% anterior cIPL is defined using the Eickhoff–Zilles PGp probabilistic
% cytoarchitectonic map
In contrast, the aPPA responds more to high-level features of scenes (e.g.
real-word size \citep{park2015parametric}; a scene's abstract category or
context\citep{marchette2015outside, watson2016patterns}) and objects (e.g.
spatial contextual associations \citep{aminoff2007parahippocampal,
aminoff2013role}) than the pPPA.
% connectivity
Moreover, pPPA and aPPA show differences in connectivity profiles.
% posterior
The pPPA exhibits more coactivation with the occipital visual cortex than the
aPPA \citep{baldassano2013differential, baldassano2016two}.
% including lateral occipital cortex (LOC) and transverse occipital sulcus (TOS)
% anterior
Activity in the aPPA, on the other hand, is found to be correlated with
components of the default mode network, including caudal inferior parietal lobe,
retrosplenial complex, medial prefrontal cortex, and lateral surface of the
anterior temporal lobe \citep{baldassano2013differential, baldassano2016two}.
% conclusion: subregions
\citet{baldassano2013differential} propose that the PPA creates a complete scene
representation, based on different aspects of a visual scene processed in
subregions of the PPA.
% Their distinct connectivity properties do suggest that each may be involved in
% specific aspects of visual and cognitive processing involved in the
% overarching goal of scene understanding \citep{baldassano2013differential}.
% The fact that anterior PPA had a lower sensitivity to our abstract object
% stimuli does not necessarily imply that this region does not use object
% information \citep{baldassano2013differential}. Previous work has shown that
% PPA responds to objects that have spatial associations [Aminoff et al. 2007],
% are space-defining [Mullally and Maguire 2011], and are
% navigationally-relevant [Janzen and Van Turennout 2004]. These types of
% responses require spatial memory and cannot be based purely on visual features
% like object shape. \citep{baldassano2013differential}.  pPPA
Similarly, our results suggest that the pPPA might be more concerned with visual
spatial features that are intrinsic to pictures or movie shots of landscapes.
% aPPA
The aPPA, in contrast, might be more concerned with spatial information that
cannot only be inferred from visual stimuli but also from auditory stimuli, such
as speech.
% examples
For example, scene properties extrapolated from a context description or label,
such as ``football stadium'', ``military hospital'', or objects and their
spatial relationship, such as the descriptions ``bus stop'' and ``roadside'', or
the sound of a car passing by.

%#6 hypothesis
Based on the report by \citet{aziz2008modulation}, we hypothesized that semantic
spatial information embedded in the audio-description would correlate with
increased hemodynamic activity in the PPA.
%
Methods and results presented here differ from this previous study in key
aspects.
% diff to Aziz (2008)
\citet{aziz2008modulation} modeled events from onset to offset of sentences,
describing unknown and famous places and faces, and compared activity levels
that were averaged across voxels of regions of interest (PPA and fusiform face
area, FFA \citep{kanwisher1997ffa}) defined by a localizer experiment.
% Aziz results
Their results showed decreased activity in only the left PPA compared to
activity in the FFA for sentences describing famous places in contrast to famous
faces.
% we: model & whole brain
Here, we modeled events from onset to offset of single words and performed a
voxelwise whole-brain analysis.
% our results
Group results of the audio-description's primary contrast yielded significantly
increased hemodynamic activity spatially restricted to the anterior part of the
PPA group overlap.
% concluding statement
Contrary to \citet{aziz2008modulation}, our results suggest that auditory
spatial information compared to non-spatial information correlates with
bilaterally increased activation in the anterior part of the PPA.

%#7 previous studies: no task
It is common for conventional localizer paradigms to employ a task to keep
participants attentive to the stimuli.
% but: Epstein (1998)
Nevertheless, one early block-design study \citep{epstein1998ppa} compared
results from a paradigm that employed a perceptual judgment task of static
pictures to the same paradigm but without that task.
% results during no task
Hemodynamic activity was less but still significantly increased when
participants had no task to keep them alert and attentive to the stimuli.
% we have no task neither
The naturalistic stimulation paradigm employed here is similar in the sense that
participants had no behavioral or cognitive task (\citep[e.g., forming a mental
image of the stimuli][]{ocraven2000mental})
 but just had to ``enjoy the presentation''.
% but still: we are different
Nevertheless, the naturalistic paradigm differs from \citet{epstein1998ppa},
because the relevant stimulus features were embedded in a continuous stream of
complex auditory (and visual) information which makes it unlikely that
participants speculated on the purpose of the investigation, or performed
undesired and unknown evaluation or categorization of isolated stimuli.
% concluding statement: kinda automatic process
Our results therefore indicate that verbally communicated spatial information is
processed, in the anterior PPA, automatically and without specifically guided
attention.


% shortcomings
%#2
Our approach to movie stimulus annotation and event selection differs from
previous reports in the literature.
% PPA: intro
In an earlier study, \citet{bartels2004mapping} manually annotated the content
of movie frames (color, faces, language, and human bodies) and found that the
functional specialization of brain areas is preserved during movie watching.
%
In contrast, we aimed to exploit a cinematographic confound in the structure of
movies, where
% establishing shots vs. e.g. over-the-shoulder shots
film directors tend to establish the spatial layout of locations in earlier
shots and later focus more on detailed depictions of people and objects
\citep{brown2012cinematography, katz1991film, mascelli1998five}.
% results
The group results of the movie stimulus' primary contrast yielded a large
cluster that spans the group overlap of individual PPAs from anterior to
posterior.
% confound of higher perception
The cluster extends into more posterior, earlier visual areas could be an
indication that the temporal averaging across events suffered from insufficient
controlling for confounding visual features.
% solution
Future studies that aim to use a movie to localize visual areas in individual
participants should extensively annotate the content of frames (e.g., using the
open-source solution ``Pliers''\citep{mcnamara2017developing} for feature
extraction from a visual naturalistic stimulus).

%#8
% optimal stimulus type in vision
In the visual domain, pictures of landscapes and not pictures of landmarks or
buildings are considered to be the ``optimal'' stimulus type
\citep{epstein2008parahippocampal}.
% we did not choose se\_new and se\_old
In the audio-description's primary contrast, we did not include the categories
that contain switches from one setting to another (\texttt{se\_new} and
\texttt{se\_old}) which one might assume to contain the auditory equivalent to
pictures of landscapes.
% why
The reason was that the categories \texttt{se\_new} and \texttt{se\_old} were
heterogeneous: they rarely contained holistic (but also vague) descriptions of
landscapes (e.g.  ``[Forrest is running through the] jungle'') but mostly
landmarks or buildings, and also non-spatial hints (e.g. ``[Jenny as a]
teenager``).
% visual system => gist in milliseconds
Humans can identify the gist of a rich visual scene within the duration of a
single fixation \citep{henderson2003human}.
%
Hence, further studies might investigate if vague verbal descriptions of
landscapes lead to a different hemodynamic activity level than descriptions of
more concrete parts of a scene (e.g.``[a] beacon'', ``[a] farmhouse'').
%#5


% MIH: second time this notion is coming up in the discussion
%% we took advantage of cinematography
%We here took advantage of an idiosyncrasy of movie directing to shift from
%depicting the spatial layout to to later depicting persons and objects.
%% know your stimulus
%Given that this shift is common in movies, investigators should be aware that
%cinematographic tendencies are part of movie's confound structure that might,
%depending on the research question, influence the results.
%% conclusion
%In summary, results from the visual localizer paradigm using blocks of pictures
%generalize to an ad hoc approach to operationalize the perception of spatial
%information at the moment of cuts in an audio-visual movie.

%#12 RSC + LOC
Apart from the PPA, results show significantly increased activity in the ventral
precuneus and posterior cingulate region (referred to as ``retrosplenial
complex'', RSC) of the medial parietal cortex, and in the superior lateral
occipital cortex (referred to as ``occipital place area'', OPA) for both
naturalistic stimuli.
% RSC intro
Like the PPA, the RSC and OPA have repeatedly  shown increased hemodynamic
activity in studies investigating visual spatial perception and navigation
\citep{chrastil2018heterogeneity, bettencourt2013role, dilks2013occipital,
epstein2019scene}).
% inference
Thus, our model-driven approach to operationalize spatial perception based on
stimulus annotations reveals increased hemodynamic responses in a network that
is implicated in visual spatial perception and cognition.
% medial parietal cortex: anterior-posterior gradient
Similarly to the parahippocampal cortex \citep{aminoff2013role}, the medial
parietal cortex exhibits a posterior-anterior gradient from being more involved
in perceptual processes to being more involved in memory related processes
\citep{chrastil2018heterogeneity, hassabis2009construction, silson2019posterior,
steel2021network}.
% future studies
Future, complementary studies using specifically designed paradigms could
investigate where in the posterior-anterior axis of the parahippocampal and
medial parietal cortex auditory semantic information is correlated with
increased hemodynamic activity:
% we hypothesize
we hypothesize that the auditory perception of spatial information (compared to
non-spatial information) is correlating with clusters in the middle of possibly
overlapping clusters correlating with visual perception (peak activity more
posterior) and scene construction from memory (peak activity more anterior).

%#13 natural stimulation
In summary, natural stimuli like movies \citep{eickhoff2020towards,
hasson2008neurocinematics, sonkusare2019naturalistic} or narratives
\citep{hamilton2018revolution, honey2012not, lerner2011topographic,
silbert2014coupled, wilson2008beyond} can be used as a continuous, complex,
immersive, task-free paradigm that more closely resembles our natural dynamic
environment than traditional experimental paradigms.
% method
We took advantage of three fMRI acquisitions and two stimulus annotations that are
part of the open-data resource
\href{http://www.studyforrest.org}{studyforrest.org} to operationalize the
perception of spatial information embedded in an audio-visual movie and an
auditory narrative, and compare current results to a previous report of a
conventional, block-design localizer.
% results
The current study offers evidence that a model-driven GLM analysis based on
annotations can be applied to a naturalistic paradigm to localize concise
functional areas and networks correlating with specific perceptual processes
-- an analysis approach that can be facilitated by the neuroscout.org platform
\citep{delavega2021neuroscout}.
% interpretation
More specifically, our results demonstrate that increased activation in the PPA
during the perception of static pictures generalizes to the perception of
spatial information embedded in a movie and an exclusively auditory stimulus.
% interpretation: aPPA vs. pPPA
Our results provide further evidence that the PPA can be divided into
functional subregions that coactivate during the perception of visual scenes.
% interpretation
Finally, the presented evidence on the in-principle suitability of a naturally
engaging, purely auditory paradigm for localizing the PPA may offer a path to
the development of diagnostic procedures more suitable for individuals with
visual impairments or conditions like nystagmus.


\section*{Methods}

% Specific data outputs should be explicitly referenced via data
% citation (see Data Records and Data Citations, below)}

% intro
We used components of the publicly available
\href{http://www.studyforrest.org}{studyforrest.org} dataset that has
been repeatedly used by other research groups in independent studies
(\citep[e.g.,][]{ben2018hippocampal, jiahui2019predicting, hu2017decoding,
lettieri2019emotionotopy, nguyen2016integration}).
% used studies
The same participants were
% AD
a) listening to the audio-description \citep{hanke2014audiomovie} of
the movie ``Forrest Gump'',
% AV
b) watching the audio-visual movie \citep{hanke2016simultaneous}, and
% VIS
c) participating in a dedicated six-category block-design visual localizer \citep{sengupta2016extension}.
% see corresponding papers for details
An exhaustive description of the participants, stimulus creation, procedure,
stimulation setup, and fMRI acquisition can be found in the corresponding
publications. Following is a summary of the most important aspects.


\subsection*{Participants}
% AD study
In the audio-description study \citep{hanke2014audiomovie}, 20 German native
speakers (all right-handed, age 21–38 years, mean age 26.6 years, 12 male)
listened to the German audio-description \citep{ForrestGumpGermanAD} of the
movie ``Forrest Gump'' \citep{ForrestGumpMovie}.
% AV study
In the movie study \citep{hanke2016simultaneous}, 15 participants (21–39 years,
mean age 29.4, six female), a subgroup of the prior audio-description study,
watched the audio-visual movie with dubbed German audio track
\citep{ForrestGumpDVD}.
% VIS study
In the block-design localizer study \citep{sengupta2016extension}, the same 15
participants took part in a six-category block-design visual localizer.
% participants' health
All participants reported to have normal hearing, normal or corrected-to-normal
vision, and no known history of neurological disorders.
% compensation, consent and shit
In all studies, participants received monetary compensation and gave written
informed consent for their participation and for public sharing of obtained data
in anonymized form. The studies had prior approval by the Ethics Committee of
Otto-von-Guericke University of Magdeburg, Germany.


\subsection*{Stimuli and Procedure}

% AD & AV stimulus name & references
The German DVD release \citep{ForrestGumpDVD} of the movie ``Forrest Gump''
\citep{ForrestGumpMovie} and its temporally aligned audio-description
\citep{ForrestGumpGermanAD} served as naturalistic stimuli, with an approximate
duration of two hours, split into eight consecutive segments of
\unit[$\approx$15]{minutes}.
% AD: additional narrator
The audio-description adds another male narrator to the voice-over narration of
the main character Forrest Gump. This additional narration describes essential
aspects of the visual scenery when there is no off-screen voice, dialog, or
other relevant auditory content.
% task
For all sessions with naturalistic stimuli, participants were instructed to
inhibit physical movements except for eye-movements, and otherwise to simply
``enjoy the presentation''.
%
For details on stimulus creation and presentation see
\citet{hanke2014audiomovie, hanke2016simultaneous}.

% VIS study picture categories
Stimuli for the block-design localizer study were 24 unique grayscale images of
faces, bodies, objects, houses, outdoor scenes and scrambled images, matched in
luminance and size, that were previously used in other studies
(\citep[e.g.,][]{haxby2011common}).
% procedure: presentation & instructions
Participants performed a one-back image matching task for four block-design
runs, with two \unit[16]{s} blocks per stimulus category in each run.
%
For details on stimulus creation and presentation see
\citet{sengupta2016extension}.


\subsection*{Stimulation setup}

% AD
In the audio-description study, visual instructions were presented on a
rear-projection screen inside the scanner bore. During the functional scans, the
projector presented a medium gray screen with the primary purpose to illuminate
a participant's visual field in order to prevent premature fatigue.
% AV & VIS
In the movie and block-design localizer study, visual instructions and stimuli
were presented on a rear-projection screen
% screen size \unit[23.75 $\times$ 10.25]{cm}
at a viewing distance of \unit[63]{cm}, with a movie frame projection size of
approximately \unit[21.3]$^{\circ}$ $\times$ \unit[9.3]$^{\circ}$.
% angle of view: VIS
In the block-design localizer study, stimulus images were displayed at a size of
approximately \unit[10]$^{\circ}$ $\times$ \unit[10]$^{\circ}$ of visual angle.
% AD & AV: auditory stimulation
Auditory stimulation was implemented using custom in-ear (audio-description), or
over-the-ear headphones (movie), which reduced the scanner noise by at least
\unit[20–30]{dB}.


\subsection*{fMRI data acquisition}

Gradient-echo fMRI data for the audio-description study were acquired using a
\unit[7]{Tesla} Siemens MAGNETOM magnetic resonance scanner equipped with a 32
channel brain receive coil at \unit[2]{s} repetition time (TR) with 36 axial
slices (thickness \unit[1.4]{mm}, \unit[1.4 $\times$ 1.4]{mm} in-plane
resolution, \unit[224]{mm} field-of-view, anterior-to-posterior phase encoding
direction) and a \unit[10]{\%} inter-slice gap, recorded in ascending order.
% slice orientation
Slices were oriented to include the ventral portions of frontal and occipital
cortex while minimizing intersection with the eyeballs.
% FOV
The field of view was centered on the approximate location of Heschl's gyrus.
% motion correction
EPI images were online-corrected for motion and geometric distortions.

% AV & VIS
In the movie and block-design localizer study, a \unit[3]{Tesla} Philips Achieva dStream
MRI scanner with a 32 channel head coil acquired gradient-echo fMRI data
at \unit[2]{s} repetition time with
% slices
35 axial slices (thickness \unit[3.0]{mm}, \unit[10]{\%} inter-slice gap) with
\unit[80 $\times$ 80]{voxels} (\unit[3.0 $\times$ 3.0]{mm} of in-plane
resolution, \unit[240]{mm} field-of-view) and an anterior-to-posterior phase
encoding direction, recorded in ascending order.
% no. of volumes
A total of 3599 volumes were recorded for each participant in each of the
naturalistic stimulus paradigms (audio-description and movie).

\subsection*{Preprocessing}

% data sources
The current analyses were carried out on the same preprocessed fMRI data
\citep{hanke2016aligned} that were used for the technical validation analysis
presented in \citet{hanke2016simultaneous}.
% exclusion of VP 10
Of those 15 participants in the studyforrest dataset that took part in all three
experiments, data of one participant were dropped due to invalid distortion
correction during scanning of the audio-description stimulus.
% preprocessing of pre-aligned data
Data were corrected for motion, aligned with and re-sliced onto a
participant-specific BOLD template image \citep{sengupta2016extension} (uniform
spatial resolution of \unit[2.5$\times$2.5$\times$2.5]{mm} for both
audio-description and movie data).
% preprocessing intro
Preprocessing was performed by FEAT v6.00 (FMRI Expert Analysis Tool
\citep{woolrich2001autocorr}) as shipped with FSL v5.0.9
(\href{https://www.fmrib.ox.ac.uk/fsl}{FMRIB's Software Library}
\citep{smith2004fsl}) on a computer-cluster running
\href{http://neuro.debian.net}{NeuroDebian} \citep{halchenko2012open}.

% temporal filtering
For the present analysis, the following additional preprocessing was performed.
High-pass temporal filtering was applied to every stimulus segment using a
Gaussian-weighted least-squares straight line with a cutoff period of
\unit[150]{s} (sigma=\unit[75.0]{s}) to remove low-frequency confounds.
% brain extraction
The brain was extracted from surrounding tissues using BET \citep{smith2002bet}.
% spatial smoothing
Data were spatially smoothed applying a Gaussian kernel with full width at half
maximum (FWHM) of \unit[4.0]{mm}.
% normalization
A grand-mean intensity normalization of the entire 4D dataset was performed by a
single multiplicative factor.
% pre-whithening
Correction for local autocorrelation in the time series (prewhitening) was
applied using FILM (FMRIB's Improved Linear Model \citep{woolrich2001autocorr})
to improve estimation efficiency.


\subsection*{Event selection}

% traditionally
In contrast to stimuli designed to trigger a perceptual process of interest,
while controlling for confounding variables (e.g., color and luminance),
% our approach: annotation and regressors
naturalistic stimuli have a fixed but initially unknown temporal structure of
stimulus features of interest, as well as an equally unknown confound structure.
% annotation ftw
In order to evaluate the suitability of the stimulus for the targeted analyses,
and to inform the required hemodynamic models, we annotated the temporal
structure of a range of stimulus features.

% AV anno
For the analysis of the movie stimulus, we took advantage of a previously
published annotation of 869 movie cuts and the depicted location after each cut
\citep{haeusler2016cutanno}.
% we focus on cuts
Contrary to manually annotating stimulus features of movie frames (for example,
as performed by \citet{bartels2004mapping} for color, faces, language, and human
bodies), we categorized movie cuts that - in general - realign the viewer within
the movie environment by switching to another perspective within the same
setting, or to a position in an entirely different setting.
% this this should work
More specifically, we sought to exploit a cinematographic bias as
% establishing shots & shots within setting early in the movie
at a setting's first occurrence in a movie, shots tend to broadly establish the
setting and the spatial layout within the setting.
% cut to recurrent scene
On revisiting an already established setting, the shot sizes tend to decrease
and more often depict people talking to each other or objects that are relevant
to the evolved plot
\citep{brown2012cinematography, katz1991film, mascelli1998five}.

% AV regressors/events
Based on this cinematographic bias, we assigned each cut to one of five
categories (see Table~\ref{tab:events}):
%
1) a cut switching to a setting that was depicted for the first time
(\texttt{vse\_new}),
%
2) a cut switching to a setting that was already depicted earlier in the movie
(\texttt{vse\_old}),
%
3) a cut switching to another locale within a setting (\texttt{vlo\_ch}; e.g. a
cut from the first to the second floor in Forrest's house),
%
4) a cut to another camera position within a setting or locale that was depicted
for the first time (\texttt{vpe\_new}), and
%
5) a cut to another camera position within a setting or locale that was already
depicted before (\texttt{vpe\_old}).
% irrespective of visual content
Note that this categorization is not necessarily evident from the visual content
of the first movie frame after a cut.
% neuroscientific rationale The neuroscientific rationale was that activity in
% the PPA is greater when participants view novel versus repeated scenes or
% view-points \citep{epstein1999parahippocampal, grill2006repetition}.  no cut
% condition: intro
As a control condition with events of no particular processing of spatial
information (\texttt{no\_cut}) we pseudo-randomly selected movie frames from
continuous movie shots that lasted longer than \unit[20]{s}, and
% no cuts: how
had a minimum temporal distance of at least \unit[10]{s} to any movie cut and to
any other \texttt{no\_cut} event.


% AV regressors
\begin{table*}[tbp]
    \caption{Overview of event categories of the audio-visual
    movie and the audio-description.
    Event categories of the movie are based on an annotation of cuts and
    depicted locations.
    Event categories of the audio-description are based on an annotation of
    nouns spoken by the audio-description's narrator
    (see Table~\ref{tab:descr-nouns-rules}).
    Some of the audio-description's event categories listed here
    (\texttt{sex\_f}; \texttt{sex\_m}; \texttt{fahead}, \texttt{object})
    were created by pooling some categories of the original annotation of nouns
    (female, females, fname; male, males, mname; face, head; object, objects).
    Respective event counts are given for the whole stimulus (\texttt{All}) and
    the segments that were used for the eight sessions of fMRI scanning.
    Event counts for frame-based features are reported in units of a thousand.
    %\texttt{fg\_av\_ger\_pd} (perceptual differences of consecutive frames),
    %\texttt{fg\_av\_ger\_ml} (mean luminance of a frame),
    %\texttt{fg\_av\_ger\_lr} (difference in mean luminance of left and right
    %half),
    %\texttt{fg\_av\_ger\_ud} (difference in mean luminance of upper and
    %lower half),
    %\texttt{fg\_av\_ger\_lrdiff} (left-right volume difference), and
    %\texttt{fg\_av\_ger\_rms} (root mean square volume) represent one event
    %for every movie frame (\unit[40]{ms}).
    % For description of the two event categories of the audio-description's
    % narrator (\texttt{se\_new} and \texttt{se\_old}) to build contrast of
    % negative control Tab.~\ref{tab:events}.
    }
\label{tab:events}
\resizebox{\linewidth}{!}{
\footnotesize
\begin{tabular}{lp{3.7cm}lllllllll} \toprule \textbf{Label} & \textbf{Description} & \textbf{All} & \textbf{1} & \textbf{2} & \textbf{3} & \textbf{4} & \textbf{5} & \textbf{6} & \textbf{7} & \textbf{8} \\
\midrule
\multicolumn{3}{l}{\textit{Movie stimulus}}\\
\texttt{vse\_new} &  change of the camera position to a setting not depicted before & \aoVsenewAll & \aoVsenewI & \aoVsenewII & \aoVsenewIII & \aoVsenewIV & \aoVsenewV & \aoVsenewVI & \aoVsenewVII & \aoVsenewVIII
\tabularnewline
\texttt{vse\_old} & change of the camera position to a recurring setting & \aoVseoldAll & \aoVseoldI & \aoVseoldII & \aoVseoldIII & \aoVseoldIV & \aoVseoldV & \aoVseoldVI & \aoVseoldVII & \aoVseoldVIII
\tabularnewline
\texttt{vlo\_ch} & change of the camera position to another locale within the same setting & \aoVlochAll & \aoVlochI & \aoVlochII & \aoVlochIII & \aoVlochIV & 0 & \aoVlochV & \aoVlochVI & \aoVlochVII
\tabularnewline
\texttt{vpe\_new} & change of the camera position within a locale not depicted before & \aoVpenewAll & \aoVpenewI & \aoVpenewII & \aoVpenewIII & \aoVpenewIV & \aoVpenewV & \aoVpenewVI & \aoVpenewVII & \aoVpenewVIII
\tabularnewline
% vpe_old has no events in 3. So the indices are shifted
\texttt{vpe\_old} & change of the camera position within a recurring locale &
\aoVpeoldAll & \aoVpeoldI & \aoVpeoldII & 0 & \aoVpeoldIII & \aoVpeoldIV &
\aoVpeoldV & \aoVpeoldVI & \aoVpeoldVII
\tabularnewline
\texttt{vno\_cut} & frames within a continuous movie shot & \avVnocutAll & \avVnocutI & \avVnocutII & 0 & \avVnocutIII & \avVnocutIV & \avVnocutV & \avVnocutVI & \avVnocutVII
\tabularnewline
% se\_new & control for AD narrator & \aoSenewAll & \aoSenewI & \aoSenewII & \aoSenewIII & \aoSenewIV & \aoSenewV & \aoSenewVI & \aoSenewVII & \aoSenewVIII
% \tabularnewline
% se\_old & control for AD narrator & \aoSeoldAll & \aoSeoldI & \aoSeoldII & \aoSeoldIII & \aoSeoldIV & \aoSeoldV & \aoSeoldVI & \aoSeoldVII & \aoSeoldVIII
% \tabularnewline
\texttt{fg\_av\_ger\_lr} & left-right luminance difference & \avFgavgerlrAll & \avFgavgerlrI & \avFgavgerlrII & \avFgavgerlrIII & \avFgavgerlrIV & \avFgavgerlrV & \avFgavgerlrVI & \avFgavgerlrVII & \avFgavgerlrVIII
\tabularnewline
\texttt{fg\_av\_ger\_lrdiff} & left-right volume difference & \avFgavgerlrdiffAll & \avFgavgerlrdiffI & \avFgavgerlrdiffII & \avFgavgerlrdiffIII & \avFgavgerlrdiffIV & \avFgavgerlrdiffV & \avFgavgerlrdiffVI & \avFgavgerlrdiffVII & \avFgavgerlrdiffVIII
\tabularnewline
\texttt{fg\_av\_ger\_ml} & mean luminance & \avFgavgermlAll & \avFgavgermlI & \avFgavgermlII & \avFgavgermlIII & \avFgavgermlIV & \avFgavgermlV & \avFgavgermlVI & \avFgavgermlVII & \avFgavgermlVIII
\tabularnewline
\texttt{fg\_av\_ger\_pd} & perceptual difference & \avFgavgerpdAll & \avFgavgerpdI & \avFgavgerpdII & \avFgavgerpdIII & \avFgavgerpdIV & \avFgavgerpdV & \avFgavgerpdVI & \avFgavgerpdVII & \avFgavgerpdVIII
\tabularnewline
\texttt{fg\_av\_ger\_rms} & root mean square volume & \avFgavgerrmsAll & \avFgavgerrmsI & \avFgavgerrmsII & \avFgavgerrmsIII & \avFgavgerrmsIV & \avFgavgerrmsV & \avFgavgerrmsVI & \avFgavgerrmsVII & \avFgavgerrmsVIII
\tabularnewline
\texttt{fg\_av\_ger\_ud} & upper-lower luminance difference & \avFgavgerudAll & \avFgavgerudI & \avFgavgerudII & \avFgavgerudIII & \avFgavgerudIV & \avFgavgerudV & \avFgavgerudVI & \avFgavgerudVII & \avFgavgerudVIII
\tabularnewline
\midrule
\multicolumn{3}{l}{\textit{Audio-description stimulus}}\\
\texttt{body} & trunk of the body; overlaid clothes & \aoBodyAll & \aoBodyI & \aoBodyII
& \aoBodyIII & \aoBodyIV & \aoBodyV & \aoBodyVI & \aoBodyVII & \aoBodyVIII
\tabularnewline
\texttt{bpart} & limbs and trousers & \aoBpartAll & \aoBpartI & \aoBpartII & \aoBpartIII & \aoBpartIV & \aoBpartV & \aoBpartVI & \aoBpartVII & \aoBpartVIII
\tabularnewline
\texttt{fahead} & face or head (parts) & \aoFaheadAll & \aoFaheadI & \aoFaheadII & \aoFaheadIII & \aoFaheadIV & \aoFaheadV & \aoFaheadVI & \aoFaheadVII & \aoFaheadVIII
\tabularnewline
\texttt{furn} & moveable furniture (insides \& outsides) & \aoFurnAll & \aoFurnI & \aoFurnII & \aoFurnIII & \aoFurnIV & \aoFurnV & \aoFurnVI & \aoFurnVII & \aoFurnVIII
\tabularnewline
\texttt{geo} & immobile landmarks & \aoGeoAll & \aoGeoI & \aoGeoII & \aoGeoIII & \aoGeoIV & \aoGeoV & \aoGeoVI & \aoGeoVII & \aoGeoVIII
\tabularnewline
\texttt{groom} & rooms \& locales or geometry-defining elements & \aoGroomAll & \aoGroomI & \aoGroomII & \aoGroomIII & \aoGroomIV & \aoGroomV & \aoGroomVI & \aoGroomVII & \aoGroomVIII
\tabularnewline
\texttt{object} & countable entities with firm boundaries & \aoObjAll & \aoObjI & \aoObjII & \aoObjIII & \aoObjIV & \aoObjV & \aoObjVI & \aoObjVII & \aoObjVIII
\tabularnewline
\texttt{se\_new} & a setting occurring for the first time & \aoSenewAll & \aoSenewI & \aoSenewII & \aoSenewIII & \aoSenewIV & \aoSenewV & \aoSenewVI & \aoSenewVII & \aoSenewVIII
\tabularnewline
\texttt{se\_old} & a recurring setting & \aoSeoldAll & \aoSeoldI & \aoSeoldII & \aoSeoldIII & \aoSeoldIV & \aoSeoldV & \aoSeoldVI & \aoSeoldVII & \aoSeoldVIII
\tabularnewline
\texttt{sex\_f} & female person(s), name & \aoSexfAll & \aoSexfI & \aoSexfII & \aoSexfIII & \aoSexfIV & \aoSexfV & \aoSexfVI & \aoSexfVII & \aoSexfVIII
\tabularnewline
\texttt{sex\_m} & male person(s), name & \aoSexmAll & \aoSexmI & \aoSexmII & \aoSexmIII & \aoSexmIV & \aoSexmV & \aoSexmVI & \aoSexmVII & \aoSexmVIII
\tabularnewline
% vlo_ch has no events in segment 5. So indices are shifted
%vse\_new & control for movie cut & \aoVsenewAll & \aoVsenewI & \aoVsenewII & \aoVsenewIII & \aoVsenewIV & \aoVsenewV & \aoVsenewVI & \aoVsenewVII & \aoVsenewVIII
%\tabularnewline
%vse\_old & control for movie cut & \aoVseoldAll & \aoVseoldI & \aoVseoldII & \aoVseoldIII & \aoVseoldIV & \aoVseoldV & \aoVseoldVI & \aoVseoldVII & \aoVseoldVIII
%\tabularnewline
%vlo\_ch & control for movie cut & \aoVlochAll & \aoVlochI & \aoVlochII & \aoVlochIII & \aoVlochIV & 0 & \aoVlochV & \aoVlochVI & \aoVlochVII
%\tabularnewline
%vpe\_new & control for movie cut & \aoVpenewAll & \aoVpenewI & \aoVpenewII & \aoVpenewIII & \aoVpenewIV & \aoVpenewV & \aoVpenewVI & \aoVpenewVII & \aoVpenewVIII
%\tabularnewline
% vpe_old has no events in 3. So the indices are shifted
%vpe\_old & control for movie cut & \aoVpeoldAll & \aoVpeoldI & \aoVpeoldII & 0 & \aoVpeoldIII & \aoVpeoldIV & \aoVpeoldV & \aoVpeoldVI & \aoVpeoldVII
% \tabularnewline
\texttt{fg\_ad\_lrdiff} & left-right volume difference & \aoFgadlrdiffAll & \aoFgadlrdiffI & \aoFgadlrdiffII & \aoFgadlrdiffIII & \aoFgadlrdiffIV &
\aoFgadlrdiffV & \aoFgadlrdiffVI & \aoFgadlrdiffVII & \aoFgadlrdiffVIII
\tabularnewline
\texttt{fg\_ad\_rms} & root mean square volume & \aoFgadrmsAll &
\aoFgadrmsI & \aoFgadrmsII & \aoFgadrmsIII & \aoFgadrmsIV & \aoFgadrmsV &
\aoFgadrmsVI & \aoFgadrmsVII & \aoFgadrmsVIII
\tabularnewline
\bottomrule
\end{tabular}
}%resizebox
\end{table*}


% AD annotation
For the analysis of the audio-description stimulus, we extended a publicly
available annotation of its speech content \citep{haeusler2021studyforrest} by
classifying concrete and countable nouns that the narrator uses to describe the
movie's absent visual content.
% annotation procedure
An initial annotation was performed by one individual,
% corrections
and minor corrections were applied after comparing with a second categorization
done by the author.
% reference to table with rules and examples
A complete overview of all 18 noun categories, their inclusion criteria, and
examples can be seen in Table~\ref{tab:descr-nouns-rules}.
% why and how of categories 1
Some categories reflect the verbal counterpart of the stimulus categories that
were used in the visual localizer experiment (e.g. \texttt{body}, \texttt{face},
\texttt{head}, \texttt{object}, \texttt{setting\_new}, and
\texttt{setting\_rec}).
% why and how of categories 2
Other categories were created to semantically cluster remaining nouns into
categories that had no counterpart in the visual localizer experiment (e.g.
\texttt{bodypart}, \texttt{female}, \texttt{fname}, \texttt{furniture},
\texttt{geo}, \texttt{groom}, \texttt{male}, and \texttt{persons}).
% categories in detail
%Nouns were categorized by the verbal clue they provide about the
%cinematographic scene's environment (\texttt{geo}, \texttt{groom};
%\texttt{setting\_new}, \texttt{setting\_old}), its inherent persons (e.g.
%\texttt{female}, \texttt{male}, \texttt{persons}), a person's body or worn
%clothes (\texttt{face}, \texttt{head}, \texttt{body}, \texttt{bodypart}), and a
%scene's inherent objects (\texttt{object}, \texttt{furniture}).  problem with
%hierarchical categories
The categories \texttt{setting\_new} and \texttt{setting\_rec} comprise not just
words that describe a setting as a whole (e.g. ``[in] Greenbow'', ``[in a]
disco'', ``[the platoon wades through a] rice field''). They also comprise words
that could count as a member of another category in case the narrator uses these
words to indicate a switch from one setting to another (e.g. ``[a] physician'').
% how it was handled
These nouns were flagged with both categories during the initial annotation
procedure. In context of the current analyses these word are considered to
belong exclusively to the higher-level category of changing a setting
(\texttt{setting\_new} or \texttt{setting\_rec}).
% pooling of categories
Some noun categories that were semantically similar and offered only a small
amount of counts were pooled resulting in 11 final event categories
(see Table~\ref{tab:events}).  These event categories were then used to model
hemodynamic responses.


% table for descriptive nouns: categories, rules, examples counts
\begin{table*}[tbp]
    \caption{Categories and criteria to categorize the nouns spoken by
        the audio-description's narrator.
        Examples are given in English.
        Some of these initial 18 noun categories were pooled resulting in 11
        event categories that served as basis to build the regressors of the
        GLM
        (see Table~\ref{tab:events}).
        % The category \texttt{++} also contains adverbial of time.
}
\label{tab:descr-nouns-rules}
\resizebox{\linewidth}{!}{
\begin{tabular}{lp{61mm}p{61mm}}
\toprule
\textbf{Category} & \textbf{Criteria} & \textbf{Examples} \\
\midrule
\texttt{body} & trunk of the body; possibly clothed & back, hip, shoulder; jacket, dress, shirt
\tabularnewline
\texttt{bodypart} & limbs & arm, finger, leg, toe
\tabularnewline
\texttt{face} & face or parts of it & face, ear, nose, mouth
\tabularnewline
\texttt{female} & female person & nurse, mother, woman
\tabularnewline
\texttt{females} & female persons & women
\tabularnewline
\texttt{fname} & female name & Jenny
\tabularnewline
\texttt{furniture} & movable furniture (insides \& outsides) & bench, bed, table, chair
\tabularnewline
\texttt{geo} & immobile landmarks & building, tree, street, alley, meadow, cornfield
\tabularnewline
\texttt{groom} & rooms \& locales, or geometry-defining elements & living room; wall, door, window, floor
\tabularnewline
\texttt{head} & non-face parts of the head; worn headgear & head, hair, ear, neck;
helmet
\tabularnewline
\texttt{male} & male person & man, father, soldier
\tabularnewline
\texttt{males} & male persons & boys, opponents
\tabularnewline
\texttt{mname} & male name & Bubba, Kennedy
\tabularnewline
\texttt{object} & countable entity with firm boundaries & telephone, car
\tabularnewline
\texttt{objects} & countable entities & wheels, plants
\tabularnewline
\texttt{persons} & concrete persons of unknown sex & hippies, patients
\tabularnewline
\texttt{setting\_new} & a setting occurring for the first time & on a ``bridge'', on an ``alley'', on ``campus''
\tabularnewline
\texttt{setting\_rec} & a recurring setting & at the ``bus stop'' \tabularnewline
% ++ & cue regarding time & in the ``evening'', it's ``daytime'', ``later'' \tabularnewline
\bottomrule
\end{tabular}
}%resizebox
\end{table*}


% low-level confounds: intro
Lastly, we algorithmically annotated the temporal structure of low-level
perceptual features to create nuisance regressors.
% visual shit
For the movie stimulus, we computed the mean luminance (arbitrary units,
average pixel brightness) of a movie frame (\unit[40]{ms}), the difference in
mean luminance of each frame's left and right half, the difference in mean
luminance of the lower and upper half. As an indicator of visual change, we
computed perceptual fingerprints for each movie frame using the pHash library
\citep{zauner2010implementation} and recorded the cumulative bitwise
difference to the previous frame.
% auditory shit
Finally for both audio tracks, we computed the left-right difference in volume
and root mean square volume averaged across the length of every movie frame
(\unit[40]{ms}).


\subsection*{Hemodynamic Modeling}

% movie events
The events of the movie cut related categories and \texttt{no\_cut} events were
modeled as box car events of \unit[200]{ms} duration,
%
speech events from onset to offset of each word, and frame-based confounds from
onset to offset of the corresponding movie frame.
% convolving
For all regressors, the stimulus models were convolved with a double-gamma
hemodynamic response function (HRF), as implemented in FSL.

% we checked correlations
Given the unknown confound structure of the naturalistic stimuli, we inspected
% correlations within stimuli and across stimuli
the correlation of regressors of each stimulus and
also across both stimuli (see Figure~\ref{fig:reg-corr}).
% correlations within each stimulus
The computed Pearson correlation coefficients show only minor correlations of
feature regressors for the same stimulus.
% (highest) correlations across stimuli; \texttt{fg\_a(v|d)\_ger\_rms}, r=.76;
% \texttt{fg\_a(v|d)\_ger\_lrdiff}, r=.77)
Maximum correlations are found for low-level auditory confounds across
stimuli,
% no surprise
which is to be expected, as the audio-description contains the audio track of
the movie plus the additional narrator (see Figure~\ref{fig:reg-corr}; correlation
between root mean square volume of both audio tracks, r=.76; correlation between
left-right volume difference of both audio tracks, r=.77).
% encouraged by that we build contrasts (statement)
The observed correlation pattern did not indicate problematic confounds of
nuisance variables and regressors of interest for any of the planned analyses.


\begin{figure*}[tbp]
\centering
    \includegraphics[width=\linewidth]{figures/regressor-corr} \caption{Pearson
        correlation coefficients of model response time series used as regressors in
        the GLM analysis of the
        audio-description (blue; see Table~\ref{tab:events} for a
        description) and audio-visual movie (red; see Table~\ref{tab:events}).
        Values are rounded to the nearest tenth.
        The correlation between the two stimuli's
        root mean square volume and between their left-right difference in
        volume yielded the highest correlation values
        (\texttt{fg\_ad\_rms} and \texttt{fg\_av\_ger\_rms}, r=.7635;
        \texttt{fg\_ad\_lrdiff} and \texttt{fg\_av\_ger\_lrdiff}, r=.7749).
      }
\label{fig:reg-corr}
\end{figure*}


% AV design matrix
The design matrix for the first-level time-series GLM analysis of the movie
comprised regressors for the 12 event categories of the movie listed in
Table~\ref{tab:events}.
% AD design matrix
Similarly, the design matrix for the analysis of the audio-description comprised
regressors for the 13 event categories of the audio-description, also listed in
Table~\ref{tab:events}.
% AD regressors in AV
In order to implement cross-modal control contrasts, the design matrix for the
movie stimulus also contained the regressors based on nouns used by the
audio-description's narrator to indicate a switch to another setting (categories
\texttt{se\_new} and \texttt{se\_old}).
% AV regressors in AV
Likewise, the design matrix for the audio-description included the five movie
cut related regressors (\texttt{vse\_new}, \texttt{vse\_old}, \texttt{vlo\_ch},
\texttt{vp\_new}, \texttt{vpe\_old}).
% both stimuli: null regressors
For both stimuli, a null regressor was used for any event category in a segment
for which no event of a category was present (e.g. no event of \texttt{vpe\_old}
in segment 3; see Table~\ref{tab:events}).
% temporal derivatives
Temporal derivatives of all regressors were added to the design matrix to
compensate for regional differences \citep{friston1998event}.
% motion parameters
Lastly, six participant-specific and run-specific motion estimates (translation
and rotation) were also included.
% high-pass filtering
The design matrices were temporally filtered with the same high-pass filter
(cutoff period of \unit[150]{s}) as the BOLD time series.


\subsection*{Statistical analysis}

% intro
We performed standard two-level GLM analyses to aggregate results across all
BOLD acquisition runs for every participant and each stimulus separately.
% level 3
Subsequently, we conducted third-level analyses to average contrast estimates
over participants for each stimulus, respectively.

% there isn't "the one road"
Given the rich naturalistic stimuli and the availability of various stimulus
feature annotations, there are several candidates for implementing contrasts to
identify hemodynamic activity congruent with the hypothesized processing of
spatial information in the PPA.
% kind of arbitrary choice
However, while the rationale for any contrast composition must be sound, the
selection of a single implementation remains arbitrary to some degree.
% hence, a couple of contrasts
Consequently, we computed multiple contrasts that are listed in
Table~\ref{tab:contrasts} for the movie and audio-description, respectively.
% subjectively assessed balance
For each stimulus, we picked a primary contrast for result presentation (marked
with an asterisk in the table) based on a subjectively assessed balance of how well
the averaged events within categories represent spatial and non-spatial
information, and the number of events in the stimulus.
% reference to supplementals
An evaluation of the robustness of these results with respect to similar but
different contrasts is provided in the Supplementary Information.

% AV: primary contrast
In the primary contrast of the movie stimulus, we contrasted cuts to a setting
that was not depicted before (\texttt{vse\_new}) to cuts merely switching the
camera position within a setting that was already established earlier in the
movie (\texttt{vpe\_old}).
% reason
We chose the contrast \texttt{vse\_new > vpe\_old} as the primary contrast,
because it is contrasting the most different categories in regard to the
averaged movie frames' content:
% explanation b
due to a cinematographic bias, the category \texttt{vse\_new} tends to comprise
mostly shots that were designed to orient the viewer in the movie's broader
environment using wider shots sizes (ranging from ``establishing shots'' to
``wide shots'').
% explanation b
In contrast, the category \texttt{vpe\_old} comprises mostly
shots depicting details of a scene using narrower shot sizes (ranging from
``medium shots'' to ``close ups'').
% AD: primary contrast In the primary contrast of the
audio-description stimulus, we contrasted mentions of buildings and landmarks
(\texttt{geo}), and rooms and geometry-defining objects (\texttt{groom}) with
nouns referring to a person's head or face (\texttt{fahead}), and other
non-geometry related categories (\texttt{body}, \texttt{bodypart},
\texttt{fahead}, \texttt{object}, \texttt{sex\_f}, \texttt{sex\_m}).
% why furniture was not used
The category ``furniture'' (\texttt{furn}) was not used as a regressor of
interest in any contrast because these nouns could be perceived as either a
geometry-defining scene element or as an isolated object.
% why se_new & se_old were not used in the primary contrasts
The categories of nouns indicating a switch to a setting that occurred for the
first time (\texttt{se\_new}) or a setting that recurred during the plot
(\texttt{se\_old}) were not included in the primary contrast because they
comprise
% heterogeneous category
a)  nouns that could be a member of another category (e.g. ``[a] physician''),
and
% richness of verbal description vs. picture
b) nouns that only vaguely identify a scene (e.g. ``[in] Greenbow'', ``[in a]
disco'', ''[black and white] film recordings'') compared to perceptually richer
descriptions.

% control contrasts
Finally, we created several control contrasts for both stimuli
(see Table~\ref{tab:contrasts}).
% AV: no cut > cut
For the movie stimulus, four contrasts were created contrasting the no-cut
regressor (\texttt{vno\_cut}) to cut-related regressors.
% AV: narrator
One contrast was created by contrasting nouns spoken by the (missing) narrator
comparing nouns
%
that indicate a switch to another setting (\texttt{se\_new} >
\texttt{se\_old}), and
%
that were moderately correlated with movie cuts indicating a
switch to another setting (see Figure~\ref{fig:reg-corr}).
% AD stimulus
For the audio-description stimulus, we created five control contrasts to test if
activation in the PPA was correlated with moments of cuts, despite the absence
of visual stimulation.

\begin{table*}[tbp]
  \caption{Computed contrasts for the analysis of the movie and the audio
    description, and their respective purpose.
    The primary contrasts are marked with an asterisk.
    \texttt{non-spatial} refers to the event categories
    \texttt{body}, \texttt{bodypart}, \texttt{fahead},
    \texttt{object}, \texttt{sex\_f}, \texttt{sex\_m}.
    An explanation of all event categories can be found in
    Table~\ref{tab:events}.
    }
\label{tab:contrasts}
\begin{tabular}{lll}
\toprule
\textbf{Nr.} &  \textbf{Contrast} & \textbf{Purpose} \\
\midrule
\multicolumn{3}{l}{\textit{Movie stimulus}}\\
1* & \texttt{vse\_new > vpe\_old} & spatial processing \tabularnewline
2 & \texttt{vse\_new, vpe\_new > vse\_old, vpe\_old} & spatial processing \tabularnewline
3 & \texttt{vse\_new > vse\_old} & spatial processing \tabularnewline
4 & \texttt{vse\_new > vse\_old, vpe\_old} & spatial processing \tabularnewline
5 & \texttt{vse\_new, vpe\_new > vpe\_old} & spatial processing \tabularnewline
6 & \texttt{vno\_cut > vse\_new} & control \tabularnewline
7 & \texttt{vno\_cut > vse\_old} & control \tabularnewline
8 & \texttt{vno\_cut > vse\_new, vse\_old} & control \tabularnewline
9 & \texttt{vno\_cut > vpe\_new, vpe\_old} & control \tabularnewline
    10 & \texttt{se\_new > se\_old} & control (absent narrator) \tabularnewline
\midrule
\multicolumn{3}{l}{\textit{Audio-description stimulus}}\\
1* & \texttt{geo, groom > non-spatial} & spatial processing \tabularnewline
2 & \texttt{geo, groom, se\_new > non-spatial} & spatial processing \tabularnewline
3 & \texttt{groom, se\_new, se\_old > non-spatial}  & spatial processing \tabularnewline
4 & \texttt{geo > non-spatial} & spatial processing \tabularnewline
5 & \texttt{groom > non-spatial} & spatial processing \tabularnewline
6 & \texttt{se\_new > non-spatial} & spatial processing \tabularnewline
7 & \texttt{se\_new, se\_old > non-spatial} & spatial processing \tabularnewline
8 & \texttt{se\_new > se\_old} & spatial processing \tabularnewline
9 & \texttt{vse\_new > vpe\_old} & control (absent visual cuts) \tabularnewline
10 & \texttt{vse\_new, vpe\_new > vse\_old, vpe\_old} & control (absent visual cuts) \tabularnewline
11 & \texttt{vse\_new > vse\_old} & control (absent visual cuts) \tabularnewline
12 & \texttt{vse\_new > vse\_old, vpe\_old} & control (absent visual cuts) \tabularnewline
13 & \texttt{vse\_new, vpe\_new > vpe\_old} & control (absent visual cuts) \tabularnewline
\bottomrule
\end{tabular}
\end{table*}


% alignments single participants: 1st level
First-level GLM analyses were performed in the image-space of a
participant-specific BOLD T2* template using a previously determined linear
transformation \citep{sengupta2016extension}.
% subject template (for Bland-Altman-Plot of unthresholded maps)
% \href{"https://github.com/psychoinformatics-de/studyforrest-data-templatetransforms/blob/master/sub-01/bold3Tp2/brain.nii.gz"}{example})
% group template (for group and individual brain slices)
For higher-level analyses image data were aligned with a study-specific T2*
group template likewise using a previously computed non-linear transformation
\citep{hanke2014audiomovie}.
% https://github.com/psychoinformatics-de/studyforrest-data-templatetransforms/blob/master/templates/grpbold3Tp2/brain.nii.gz"
% MNI152
This group template was co-registered to the MNI 152 template with an affine
transformation (12 degrees of freedom).

% second level model
The second-level analyses, which averaged contrast estimates over the eight
stimulus segments, were carried out using a fixed-effects model by forcing the
random effects variance to zero in FLAME (FMRIB's Local Analysis of Mixed
Effects \citep{beckmann2003general, woolrich2004multilevel}).
% thresholding ($Z$>2.3 in subject-space; $Z$>3.4 in group space)
(Gaussianised T/F) statistic images were thresholded using clusters determined
by $Z$>3.4 and a cluster-corrected significance threshold of $p$=.05
\citep{worsley2001statistical}.

% third level model
The third level analysis which averaged contrast estimates over participants was
carried out using FLAME stage 1 with automatic outlier detection
\citep{beckmann2003general, woolrich2004multilevel, woolrich2008robust}.
% thresholding
Here again, Z (Gaussianised T/F) statistic images were thresholded using
clusters determined by $Z$>3.4 and a corrected cluster significance threshold of
$p$=.05 \citep{worsley2001statistical}.
% brain region identification
Brain regions associated with observed clusters were determined with the Jülich
Histological Atlas \citep{eickhoff2005toolbox, eickhoff2007assignment} and the
Harvard-Oxford Cortical Atlas \citep{desikan2006automated} provided by FSL.
% PPA masks
Regions of interest masks for individual PPAs and a PPA group mask of
individual PPA overlaps were created from data provided by
\citet{sengupta2016extension}.



\section*{Data Availability}

% \href{https://gin.g-node.org/chaeusler/studyforrest-ppa-analysis}{\url{gin.g-node.org/chaeusler/studyforrest-ppa-analysis}}

% new; PPA analysis
All fMRI data, annotations, and results are available as Datalad \citep{datalad}
datasets, published to or linked from the \emph{G-Node GIN} repository
(\href{https://doi.org/10.12751/g-node.qmevq0}{\url{doi.org/10.12751/g-node.qmevq0}}\citep{haeusler2021ppa}).
% original
Raw data of the audio-description, movie and visual localizer were originally
published on the \emph{OpenfMRI} portal
(\url{https://legacy.openfmri.org/dataset/ds000113}\citep{Hanke2014ds000113},
\url{https://legacy.openfmri.org/dataset/ds000113d}\citep{hanke2016ds000113d}).
% visual localizer
Results from the localization of higher visual areas are available as Datalad
datasets at \emph{GitHub} (
\href{https://github.com/psychoinformatics-de/studyforrest-data-visualrois}{\url{github.com/psychoinformatics-de/studyforrest-data-visualrois}}\citep{sengupta2016extensiondata}).
% raw data
The realigned participant-specific timeseries that were used in the current
analyses were derived from the raw data releases and are available as Datalad
datasets at \emph{GitHub} (
\href{https://github.com/psychoinformatics-de/studyforrest-data-aligned}{\url{github.com/psychoinformatics-de/studyforrest-data-aligned}}
\citep{hanke2016aligned}).
% cut annotation
The annotation of cuts is available at \emph{F1000Research}
(\href{https://doi.org/10.5256/f1000research.9536.d134823}{\url{doi.org/10.5256/f1000research.9536.d134823}}
\citep{haeusler2016cutdata}),
% speech annotation
the annotation of speech is available at \emph{OSF}
(\href{https://doi.org/10.17605/OSF.IO/F5J3E}{\url{doi.org/10.17605/OSF.IO/F5J3E}}
\citep{haeusler2020speechdata}).
% surface data
Participant-specific reconstructed cortical surface are available as Datalad
dataset at \emph{GitHub} (
\href{https://github.com/psychoinformatics-de/studyforrest-data-freesurfer}{\url{github.com/psychoinformatics-de/studyforrest-data-freesurfer}}
\citep{hanke2016freesurferdata}), and are also part of the original data release
on \emph{OpenfMRI}
(\url{https://legacy.openfmri.org/dataset/ds000113}\citep{Hanke2014ds000113}.

The same data are available in a modified and merged form on OpenNeuro at \url{https://openneuro.org/datasets/ds000113}.
Unthresholded $Z$-maps of all contrasts can be found at
\href{https://neurovault.org/collections/KADGMGVZ/}{\url{neurovault.org/collections/KADGMGVZ}}.


\section*{Code Availability}

Scripts to generate the results as Datalad \citep{datalad} datasets are
available in a \emph{G-Node GIN} repository
(\href{https://doi.org/10.12751/g-node.qmevq0}{\url{doi.org/10.12751/g-node.qmevq0}}\citep{haeusler2021ppa}).



{\small\bibliographystyle{unsrtnat}
\bibliography{references}}


%\expandafter\ifx\csname url\endcsname\relax
%  \def\url#1{\texttt{#1}}\fi
%\expandafter\ifx\csname urlprefix\endcsname\relax\def\urlprefix{URL }\fi
%\providecommand{\bibinfo}[2]{#2}
%\providecommand{\eprint}[2][]{\url{#2}}
%
%\bibitem{cite1}
%\bibinfo{author}{Califano, A.}, \bibinfo{author}{Butte, A.~J.},
%  \bibinfo{author}{Friend, S.}, \bibinfo{author}{Ideker, T.} \&
%  \bibinfo{author}{Schadt, E.}
%\newblock \bibinfo{title}{{Leveraging models of cell regulation and GWAS data
%  in integrative network-based association studies}}.
%\newblock \emph{\bibinfo{journal}{Nature Genetics}}
%  \textbf{\bibinfo{volume}{44}}, \bibinfo{pages}{841--847}
%  (\bibinfo{year}{2012}).
%
%\bibitem{cite2}
%\bibinfo{author}{Wang, R.} \emph{et~al.}
%\newblock \bibinfo{title}{{PRIDE Inspector: a tool to visualize and validate MS
%  proteomics data.}}
%\newblock \emph{\bibinfo{journal}{Nature Biotechnology}}
%  \textbf{\bibinfo{volume}{30}}, \bibinfo{pages}{135--137}
%  (\bibinfo{year}{2012}).
%\end{thebibliography}


\section*{Acknowledgements}


We are grateful to \href{www.florianschurz.de}{Florian Schurz} who initiated the
annotation of nouns and created its preliminary version.%

This research was, in part, supported by the German Federal Ministry of
Education and Research (BMBF, 01GQ1112) and the US National Science Foundation
(NSF, 1129855). Work on the data-sharing technology employed for this research
was supported by BMBF (01GQ1411, 01GQ1905) and NSF (1429999, 1912266).


\section*{Author Contributions}

COH designed and performed the analyses, and wrote the manuscript.
%
MH provided critical feedback on the procedure and edited the manuscript.
%
SE provided critical feedback.


\section*{Competing Interests}

The authors declare no competing financial interests.




\end{document}
