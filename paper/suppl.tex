\documentclass[english,11pt]{article}
\usepackage[utf8]{inputenc}
\usepackage[T1]{fontenc}
\usepackage{stix2}
\usepackage{babel}
\usepackage[section]{placeins}
\usepackage{flafter}
\usepackage{comment}
\usepackage{breakurl}
\usepackage{svg}
\usepackage[labelfont=bf,font=small]{caption}
\usepackage{booktabs}
\usepackage{amsmath}
\usepackage{graphicx}
\usepackage{fancyhdr}
% \usepackage[disable]{todonotes}
\usepackage{todonotes}
\usepackage{units}
\usepackage[left=2.5cm,right=2.5cm,top=1cm,bottom=1cm,includeheadfoot]{geometry}
\usepackage[
colorlinks=true,
	citecolor=black,
  urlcolor=black,
	linkcolor=black
]{hyperref}
%\newcommand{\scidatalogo}{\includegraphics[height=36pt]{SciData_logo.jpg}}
\newcommand{\scidatalogo}{}
\pagestyle{fancy}
\fancyhf{}
\renewcommand{\headrulewidth}{0pt}
%\setlength{\headheight}{40pt}
%\lhead{\textsc{\scidatalogo}}
\usepackage[comma,super,sort&compress]{natbib}
\usepackage{xr}
\externaldocument{main}
\makeatletter
\renewcommand{\thefigure}{S\@arabic\c@figure}
\makeatother
% \usepackage{helvet}

% \renewcommand{\familydefault}{\sfdefault}

\begin{document}
\newcommand{\aoBodyAll}{66}
\newcommand{\aoBodyI}{6}
\newcommand{\aoBodyII}{12}
\newcommand{\aoBodyIII}{7}
\newcommand{\aoBodyIV}{12}
\newcommand{\aoBodyV}{2}
\newcommand{\aoBodyVI}{9}
\newcommand{\aoBodyVII}{13}
\newcommand{\aoBodyVIII}{5}

\newcommand{\aoBpartAll}{69}
\newcommand{\aoBpartI}{9}
\newcommand{\aoBpartII}{8}
\newcommand{\aoBpartIII}{6}
\newcommand{\aoBpartIV}{13}
\newcommand{\aoBpartV}{5}
\newcommand{\aoBpartVI}{7}
\newcommand{\aoBpartVII}{11}
\newcommand{\aoBpartVIII}{10}

\newcommand{\aoFaheadAll}{83}
\newcommand{\aoFaheadI}{12}
\newcommand{\aoFaheadII}{11}
\newcommand{\aoFaheadIII}{10}
\newcommand{\aoFaheadIV}{5}
\newcommand{\aoFaheadV}{9}
\newcommand{\aoFaheadVI}{13}
\newcommand{\aoFaheadVII}{12}
\newcommand{\aoFaheadVIII}{11}

\newcommand{\aoFgadlrdiffAll}{180k}
\newcommand{\aoFgadlrdiffI}{22k}
\newcommand{\aoFgadlrdiffII}{22k}
\newcommand{\aoFgadlrdiffIII}{21k}
\newcommand{\aoFgadlrdiffIV}{24k}
\newcommand{\aoFgadlrdiffV}{23k}
\newcommand{\aoFgadlrdiffVI}{21k}
\newcommand{\aoFgadlrdiffVII}{27k}
\newcommand{\aoFgadlrdiffVIII}{16k}

\newcommand{\aoFgadrmsAll}{180k}
\newcommand{\aoFgadrmsI}{22k}
\newcommand{\aoFgadrmsII}{22k}
\newcommand{\aoFgadrmsIII}{21k}
\newcommand{\aoFgadrmsIV}{24k}
\newcommand{\aoFgadrmsV}{23k}
\newcommand{\aoFgadrmsVI}{21k}
\newcommand{\aoFgadrmsVII}{27k}
\newcommand{\aoFgadrmsVIII}{16k}

\newcommand{\aoFurnAll}{50}
\newcommand{\aoFurnI}{8}
\newcommand{\aoFurnII}{5}
\newcommand{\aoFurnIII}{2}
\newcommand{\aoFurnIV}{5}
\newcommand{\aoFurnV}{7}
\newcommand{\aoFurnVI}{10}
\newcommand{\aoFurnVII}{7}
\newcommand{\aoFurnVIII}{6}

\newcommand{\aoGeoAll}{125}
\newcommand{\aoGeoI}{16}
\newcommand{\aoGeoII}{17}
\newcommand{\aoGeoIII}{11}
\newcommand{\aoGeoIV}{32}
\newcommand{\aoGeoV}{0}
\newcommand{\aoGeoVI}{15}
\newcommand{\aoGeoVII}{18}
\newcommand{\aoGeoVIII}{16}

\newcommand{\aoGroomAll}{105}
\newcommand{\aoGroomI}{12}
\newcommand{\aoGroomII}{11}
\newcommand{\aoGroomIII}{8}
\newcommand{\aoGroomIV}{5}
\newcommand{\aoGroomV}{8}
\newcommand{\aoGroomVI}{25}
\newcommand{\aoGroomVII}{28}
\newcommand{\aoGroomVIII}{8}

\newcommand{\aoObjAll}{284}
\newcommand{\aoObjI}{39}
\newcommand{\aoObjII}{34}
\newcommand{\aoObjIII}{27}
\newcommand{\aoObjIV}{44}
\newcommand{\aoObjV}{29}
\newcommand{\aoObjVI}{42}
\newcommand{\aoObjVII}{32}
\newcommand{\aoObjVIII}{37}

\newcommand{\aoSenewAll}{86}
\newcommand{\aoSenewI}{11}
\newcommand{\aoSenewII}{15}
\newcommand{\aoSenewIII}{12}
\newcommand{\aoSenewIV}{4}
\newcommand{\aoSenewV}{15}
\newcommand{\aoSenewVI}{10}
\newcommand{\aoSenewVII}{16}
\newcommand{\aoSenewVIII}{3}

\newcommand{\aoSeoldAll}{37}
\newcommand{\aoSeoldI}{2}
\newcommand{\aoSeoldII}{5}
\newcommand{\aoSeoldIII}{1}
\newcommand{\aoSeoldIV}{4}
\newcommand{\aoSeoldV}{2}
\newcommand{\aoSeoldVI}{9}
\newcommand{\aoSeoldVII}{8}
\newcommand{\aoSeoldVIII}{6}

\newcommand{\aoSexfAll}{108}
\newcommand{\aoSexfI}{14}
\newcommand{\aoSexfII}{22}
\newcommand{\aoSexfIII}{6}
\newcommand{\aoSexfIV}{6}
\newcommand{\aoSexfV}{13}
\newcommand{\aoSexfVI}{10}
\newcommand{\aoSexfVII}{23}
\newcommand{\aoSexfVIII}{14}

\newcommand{\aoSexmAll}{403}
\newcommand{\aoSexmI}{41}
\newcommand{\aoSexmII}{68}
\newcommand{\aoSexmIII}{38}
\newcommand{\aoSexmIV}{102}
\newcommand{\aoSexmV}{45}
\newcommand{\aoSexmVI}{42}
\newcommand{\aoSexmVII}{42}
\newcommand{\aoSexmVIII}{25}

\newcommand{\aoSexuAll}{17}
\newcommand{\aoSexuI}{0}
\newcommand{\aoSexuII}{3}
\newcommand{\aoSexuIII}{1}
\newcommand{\aoSexuIV}{2}
\newcommand{\aoSexuV}{3}
\newcommand{\aoSexuVI}{1}
\newcommand{\aoSexuVII}{5}
\newcommand{\aoSexuVIII}{2}

\newcommand{\aoVlochAll}{89}
\newcommand{\aoVlochI}{10}
\newcommand{\aoVlochII}{31}
\newcommand{\aoVlochIII}{2}
\newcommand{\aoVlochIV}{23}
\newcommand{\aoVlochV}{4}
\newcommand{\aoVlochVI}{18}
\newcommand{\aoVlochVII}{1}
\newcommand{\aoVnocutAll}{148}
\newcommand{\aoVnocutI}{30}
\newcommand{\aoVnocutII}{13}
\newcommand{\aoVnocutIII}{21}
\newcommand{\aoVnocutIV}{15}
\newcommand{\aoVnocutV}{27}
\newcommand{\aoVnocutVI}{9}
\newcommand{\aoVnocutVII}{17}
\newcommand{\aoVnocutVIII}{16}

\newcommand{\aoVpenewAll}{386}
\newcommand{\aoVpenewI}{31}
\newcommand{\aoVpenewII}{38}
\newcommand{\aoVpenewIII}{72}
\newcommand{\aoVpenewIV}{90}
\newcommand{\aoVpenewV}{89}
\newcommand{\aoVpenewVI}{33}
\newcommand{\aoVpenewVII}{24}
\newcommand{\aoVpenewVIII}{9}

\newcommand{\aoVpeoldAll}{208}
\newcommand{\aoVpeoldI}{25}
\newcommand{\aoVpeoldII}{61}
\newcommand{\aoVpeoldIII}{13}
\newcommand{\aoVpeoldIV}{1}
\newcommand{\aoVpeoldV}{32}
\newcommand{\aoVpeoldVI}{29}
\newcommand{\aoVpeoldVII}{47}
\newcommand{\aoVsenewAll}{96}
\newcommand{\aoVsenewI}{11}
\newcommand{\aoVsenewII}{14}
\newcommand{\aoVsenewIII}{17}
\newcommand{\aoVsenewIV}{4}
\newcommand{\aoVsenewV}{17}
\newcommand{\aoVsenewVI}{9}
\newcommand{\aoVsenewVII}{21}
\newcommand{\aoVsenewVIII}{3}

\newcommand{\aoVseoldAll}{90}
\newcommand{\aoVseoldI}{7}
\newcommand{\aoVseoldII}{11}
\newcommand{\aoVseoldIII}{3}
\newcommand{\aoVseoldIV}{7}
\newcommand{\aoVseoldV}{7}
\newcommand{\aoVseoldVI}{23}
\newcommand{\aoVseoldVII}{15}
\newcommand{\aoVseoldVIII}{17}


\newcommand{\avBodyAll}{66}
\newcommand{\avBodyI}{6}
\newcommand{\avBodyII}{12}
\newcommand{\avBodyIII}{7}
\newcommand{\avBodyIV}{12}
\newcommand{\avBodyV}{2}
\newcommand{\avBodyVI}{9}
\newcommand{\avBodyVII}{13}
\newcommand{\avBodyVIII}{5}

\newcommand{\avBpartAll}{69}
\newcommand{\avBpartI}{9}
\newcommand{\avBpartII}{8}
\newcommand{\avBpartIII}{6}
\newcommand{\avBpartIV}{13}
\newcommand{\avBpartV}{5}
\newcommand{\avBpartVI}{7}
\newcommand{\avBpartVII}{11}
\newcommand{\avBpartVIII}{10}

\newcommand{\avFaheadAll}{83}
\newcommand{\avFaheadI}{12}
\newcommand{\avFaheadII}{11}
\newcommand{\avFaheadIII}{10}
\newcommand{\avFaheadIV}{5}
\newcommand{\avFaheadV}{9}
\newcommand{\avFaheadVI}{13}
\newcommand{\avFaheadVII}{12}
\newcommand{\avFaheadVIII}{11}

\newcommand{\avFgavgerlrAll}{180k}
\newcommand{\avFgavgerlrI}{22k}
\newcommand{\avFgavgerlrII}{22k}
\newcommand{\avFgavgerlrIII}{22k}
\newcommand{\avFgavgerlrIV}{24k}
\newcommand{\avFgavgerlrV}{23k}
\newcommand{\avFgavgerlrVI}{22k}
\newcommand{\avFgavgerlrVII}{27k}
\newcommand{\avFgavgerlrVIII}{16k}

\newcommand{\avFgavgerlrdiffAll}{180k}
\newcommand{\avFgavgerlrdiffI}{22k}
\newcommand{\avFgavgerlrdiffII}{22k}
\newcommand{\avFgavgerlrdiffIII}{22k}
\newcommand{\avFgavgerlrdiffIV}{24k}
\newcommand{\avFgavgerlrdiffV}{23k}
\newcommand{\avFgavgerlrdiffVI}{22k}
\newcommand{\avFgavgerlrdiffVII}{27k}
\newcommand{\avFgavgerlrdiffVIII}{16k}

\newcommand{\avFgavgermlAll}{180k}
\newcommand{\avFgavgermlI}{22k}
\newcommand{\avFgavgermlII}{22k}
\newcommand{\avFgavgermlIII}{22k}
\newcommand{\avFgavgermlIV}{24k}
\newcommand{\avFgavgermlV}{23k}
\newcommand{\avFgavgermlVI}{22k}
\newcommand{\avFgavgermlVII}{27k}
\newcommand{\avFgavgermlVIII}{16k}

\newcommand{\avFgavgerpdAll}{180k}
\newcommand{\avFgavgerpdI}{22k}
\newcommand{\avFgavgerpdII}{22k}
\newcommand{\avFgavgerpdIII}{22k}
\newcommand{\avFgavgerpdIV}{24k}
\newcommand{\avFgavgerpdV}{23k}
\newcommand{\avFgavgerpdVI}{22k}
\newcommand{\avFgavgerpdVII}{27k}
\newcommand{\avFgavgerpdVIII}{16k}

\newcommand{\avFgavgerrmsAll}{180k}
\newcommand{\avFgavgerrmsI}{22k}
\newcommand{\avFgavgerrmsII}{22k}
\newcommand{\avFgavgerrmsIII}{22k}
\newcommand{\avFgavgerrmsIV}{24k}
\newcommand{\avFgavgerrmsV}{23k}
\newcommand{\avFgavgerrmsVI}{22k}
\newcommand{\avFgavgerrmsVII}{27k}
\newcommand{\avFgavgerrmsVIII}{16k}

\newcommand{\avFgavgerudAll}{180k}
\newcommand{\avFgavgerudI}{22k}
\newcommand{\avFgavgerudII}{22k}
\newcommand{\avFgavgerudIII}{22k}
\newcommand{\avFgavgerudIV}{24k}
\newcommand{\avFgavgerudV}{23k}
\newcommand{\avFgavgerudVI}{22k}
\newcommand{\avFgavgerudVII}{27k}
\newcommand{\avFgavgerudVIII}{16k}

\newcommand{\avFurnAll}{50}
\newcommand{\avFurnI}{8}
\newcommand{\avFurnII}{5}
\newcommand{\avFurnIII}{2}
\newcommand{\avFurnIV}{5}
\newcommand{\avFurnV}{7}
\newcommand{\avFurnVI}{10}
\newcommand{\avFurnVII}{7}
\newcommand{\avFurnVIII}{6}

\newcommand{\avGeoAll}{125}
\newcommand{\avGeoI}{16}
\newcommand{\avGeoII}{17}
\newcommand{\avGeoIII}{11}
\newcommand{\avGeoIV}{32}
\newcommand{\avGeoV}{0}
\newcommand{\avGeoVI}{15}
\newcommand{\avGeoVII}{18}
\newcommand{\avGeoVIII}{16}

\newcommand{\avGroomAll}{105}
\newcommand{\avGroomI}{12}
\newcommand{\avGroomII}{11}
\newcommand{\avGroomIII}{8}
\newcommand{\avGroomIV}{5}
\newcommand{\avGroomV}{8}
\newcommand{\avGroomVI}{25}
\newcommand{\avGroomVII}{28}
\newcommand{\avGroomVIII}{8}

\newcommand{\avObjAll}{284}
\newcommand{\avObjI}{39}
\newcommand{\avObjII}{34}
\newcommand{\avObjIII}{27}
\newcommand{\avObjIV}{44}
\newcommand{\avObjV}{29}
\newcommand{\avObjVI}{42}
\newcommand{\avObjVII}{32}
\newcommand{\avObjVIII}{37}

\newcommand{\avSenewAll}{86}
\newcommand{\avSenewI}{11}
\newcommand{\avSenewII}{15}
\newcommand{\avSenewIII}{12}
\newcommand{\avSenewIV}{4}
\newcommand{\avSenewV}{15}
\newcommand{\avSenewVI}{10}
\newcommand{\avSenewVII}{16}
\newcommand{\avSenewVIII}{3}

\newcommand{\avSeoldAll}{37}
\newcommand{\avSeoldI}{2}
\newcommand{\avSeoldII}{5}
\newcommand{\avSeoldIII}{1}
\newcommand{\avSeoldIV}{4}
\newcommand{\avSeoldV}{2}
\newcommand{\avSeoldVI}{9}
\newcommand{\avSeoldVII}{8}
\newcommand{\avSeoldVIII}{6}

\newcommand{\avSexfAll}{108}
\newcommand{\avSexfI}{14}
\newcommand{\avSexfII}{22}
\newcommand{\avSexfIII}{6}
\newcommand{\avSexfIV}{6}
\newcommand{\avSexfV}{13}
\newcommand{\avSexfVI}{10}
\newcommand{\avSexfVII}{23}
\newcommand{\avSexfVIII}{14}

\newcommand{\avSexmAll}{403}
\newcommand{\avSexmI}{41}
\newcommand{\avSexmII}{68}
\newcommand{\avSexmIII}{38}
\newcommand{\avSexmIV}{102}
\newcommand{\avSexmV}{45}
\newcommand{\avSexmVI}{42}
\newcommand{\avSexmVII}{42}
\newcommand{\avSexmVIII}{25}

\newcommand{\avSexuAll}{17}
\newcommand{\avSexuI}{0}
\newcommand{\avSexuII}{3}
\newcommand{\avSexuIII}{1}
\newcommand{\avSexuIV}{2}
\newcommand{\avSexuV}{3}
\newcommand{\avSexuVI}{1}
\newcommand{\avSexuVII}{5}
\newcommand{\avSexuVIII}{2}

\newcommand{\avVlochAll}{89}
\newcommand{\avVlochI}{10}
\newcommand{\avVlochII}{31}
\newcommand{\avVlochIII}{2}
\newcommand{\avVlochIV}{23}
\newcommand{\avVlochV}{4}
\newcommand{\avVlochVI}{18}
\newcommand{\avVlochVII}{1}
\newcommand{\avVnocutAll}{148}
\newcommand{\avVnocutI}{30}
\newcommand{\avVnocutII}{13}
\newcommand{\avVnocutIII}{21}
\newcommand{\avVnocutIV}{15}
\newcommand{\avVnocutV}{27}
\newcommand{\avVnocutVI}{9}
\newcommand{\avVnocutVII}{17}
\newcommand{\avVnocutVIII}{16}

\newcommand{\avVpenewAll}{386}
\newcommand{\avVpenewI}{31}
\newcommand{\avVpenewII}{38}
\newcommand{\avVpenewIII}{72}
\newcommand{\avVpenewIV}{90}
\newcommand{\avVpenewV}{89}
\newcommand{\avVpenewVI}{33}
\newcommand{\avVpenewVII}{24}
\newcommand{\avVpenewVIII}{9}

\newcommand{\avVpeoldAll}{208}
\newcommand{\avVpeoldI}{25}
\newcommand{\avVpeoldII}{61}
\newcommand{\avVpeoldIII}{13}
\newcommand{\avVpeoldIV}{1}
\newcommand{\avVpeoldV}{32}
\newcommand{\avVpeoldVI}{29}
\newcommand{\avVpeoldVII}{47}
\newcommand{\avVsenewAll}{96}
\newcommand{\avVsenewI}{11}
\newcommand{\avVsenewII}{14}
\newcommand{\avVsenewIII}{17}
\newcommand{\avVsenewIV}{4}
\newcommand{\avVsenewV}{17}
\newcommand{\avVsenewVI}{9}
\newcommand{\avVsenewVII}{21}
\newcommand{\avVsenewVIII}{3}

\newcommand{\avVseoldAll}{90}
\newcommand{\avVseoldI}{7}
\newcommand{\avVseoldII}{11}
\newcommand{\avVseoldIII}{3}
\newcommand{\avVseoldIV}{7}
\newcommand{\avVseoldV}{7}
\newcommand{\avVseoldVI}{23}
\newcommand{\avVseoldVII}{15}
\newcommand{\avVseoldVIII}{17}


\newcommand{\anAll}{17}
\newcommand{\anI}{2}
\newcommand{\anII}{2}
\newcommand{\anIII}{2}
\newcommand{\anIV}{0}
\newcommand{\anV}{3}
\newcommand{\anVI}{3}
\newcommand{\anVII}{4}
\newcommand{\anVIII}{1}

\newcommand{\anBodyAll}{66}
\newcommand{\anBodyI}{6}
\newcommand{\anBodyII}{12}
\newcommand{\anBodyIII}{7}
\newcommand{\anBodyIV}{12}
\newcommand{\anBodyV}{2}
\newcommand{\anBodyVI}{9}
\newcommand{\anBodyVII}{13}
\newcommand{\anBodyVIII}{5}

\newcommand{\anBodypartAll}{69}
\newcommand{\anBodypartI}{9}
\newcommand{\anBodypartII}{8}
\newcommand{\anBodypartIII}{6}
\newcommand{\anBodypartIV}{13}
\newcommand{\anBodypartV}{5}
\newcommand{\anBodypartVI}{7}
\newcommand{\anBodypartVII}{11}
\newcommand{\anBodypartVIII}{10}

\newcommand{\anFaceAll}{47}
\newcommand{\anFaceI}{7}
\newcommand{\anFaceII}{7}
\newcommand{\anFaceIII}{6}
\newcommand{\anFaceIV}{1}
\newcommand{\anFaceV}{7}
\newcommand{\anFaceVI}{9}
\newcommand{\anFaceVII}{6}
\newcommand{\anFaceVIII}{4}

\newcommand{\anFemaleAll}{31}
\newcommand{\anFemaleI}{12}
\newcommand{\anFemaleII}{8}
\newcommand{\anFemaleIII}{0}
\newcommand{\anFemaleIV}{3}
\newcommand{\anFemaleV}{0}
\newcommand{\anFemaleVI}{3}
\newcommand{\anFemaleVII}{2}
\newcommand{\anFemaleVIII}{3}

\newcommand{\anFemalesAll}{3}
\newcommand{\anFemalesI}{0}
\newcommand{\anFemalesII}{0}
\newcommand{\anFemalesIII}{0}
\newcommand{\anFemalesIV}{2}
\newcommand{\anFemalesV}{0}
\newcommand{\anFemalesVI}{0}
\newcommand{\anFemalesVII}{1}
\newcommand{\anFemalesVIII}{0}

\newcommand{\anFnameAll}{74}
\newcommand{\anFnameI}{2}
\newcommand{\anFnameII}{14}
\newcommand{\anFnameIII}{6}
\newcommand{\anFnameIV}{1}
\newcommand{\anFnameV}{13}
\newcommand{\anFnameVI}{7}
\newcommand{\anFnameVII}{20}
\newcommand{\anFnameVIII}{11}

\newcommand{\anFurnitureAll}{50}
\newcommand{\anFurnitureI}{8}
\newcommand{\anFurnitureII}{5}
\newcommand{\anFurnitureIII}{2}
\newcommand{\anFurnitureIV}{5}
\newcommand{\anFurnitureV}{7}
\newcommand{\anFurnitureVI}{10}
\newcommand{\anFurnitureVII}{7}
\newcommand{\anFurnitureVIII}{6}

\newcommand{\anGeoAll}{125}
\newcommand{\anGeoI}{16}
\newcommand{\anGeoII}{17}
\newcommand{\anGeoIII}{11}
\newcommand{\anGeoIV}{32}
\newcommand{\anGeoV}{0}
\newcommand{\anGeoVI}{15}
\newcommand{\anGeoVII}{18}
\newcommand{\anGeoVIII}{16}

\newcommand{\anGeoroomAll}{105}
\newcommand{\anGeoroomI}{12}
\newcommand{\anGeoroomII}{11}
\newcommand{\anGeoroomIII}{8}
\newcommand{\anGeoroomIV}{5}
\newcommand{\anGeoroomV}{8}
\newcommand{\anGeoroomVI}{25}
\newcommand{\anGeoroomVII}{28}
\newcommand{\anGeoroomVIII}{8}

\newcommand{\anHeadAll}{36}
\newcommand{\anHeadI}{5}
\newcommand{\anHeadII}{4}
\newcommand{\anHeadIII}{4}
\newcommand{\anHeadIV}{4}
\newcommand{\anHeadV}{2}
\newcommand{\anHeadVI}{4}
\newcommand{\anHeadVII}{6}
\newcommand{\anHeadVIII}{7}

\newcommand{\anMaleAll}{89}
\newcommand{\anMaleI}{15}
\newcommand{\anMaleII}{18}
\newcommand{\anMaleIII}{9}
\newcommand{\anMaleIV}{18}
\newcommand{\anMaleV}{7}
\newcommand{\anMaleVI}{8}
\newcommand{\anMaleVII}{9}
\newcommand{\anMaleVIII}{5}

\newcommand{\anMalesAll}{23}
\newcommand{\anMalesI}{2}
\newcommand{\anMalesII}{11}
\newcommand{\anMalesIII}{4}
\newcommand{\anMalesIV}{3}
\newcommand{\anMalesV}{2}
\newcommand{\anMalesVI}{0}
\newcommand{\anMalesVII}{1}
\newcommand{\anMalesVIII}{0}

\newcommand{\anMnameAll}{291}
\newcommand{\anMnameI}{24}
\newcommand{\anMnameII}{39}
\newcommand{\anMnameIII}{25}
\newcommand{\anMnameIV}{81}
\newcommand{\anMnameV}{36}
\newcommand{\anMnameVI}{34}
\newcommand{\anMnameVII}{32}
\newcommand{\anMnameVIII}{20}

\newcommand{\anObjectAll}{232}
\newcommand{\anObjectI}{36}
\newcommand{\anObjectII}{22}
\newcommand{\anObjectIII}{20}
\newcommand{\anObjectIV}{30}
\newcommand{\anObjectV}{25}
\newcommand{\anObjectVI}{37}
\newcommand{\anObjectVII}{26}
\newcommand{\anObjectVIII}{36}

\newcommand{\anObjectsAll}{52}
\newcommand{\anObjectsI}{3}
\newcommand{\anObjectsII}{12}
\newcommand{\anObjectsIII}{7}
\newcommand{\anObjectsIV}{14}
\newcommand{\anObjectsV}{4}
\newcommand{\anObjectsVI}{5}
\newcommand{\anObjectsVII}{6}
\newcommand{\anObjectsVIII}{1}

\newcommand{\anPersonsAll}{17}
\newcommand{\anPersonsI}{0}
\newcommand{\anPersonsII}{3}
\newcommand{\anPersonsIII}{1}
\newcommand{\anPersonsIV}{2}
\newcommand{\anPersonsV}{3}
\newcommand{\anPersonsVI}{1}
\newcommand{\anPersonsVII}{5}
\newcommand{\anPersonsVIII}{2}

\newcommand{\anSettingnewAll}{86}
\newcommand{\anSettingnewI}{11}
\newcommand{\anSettingnewII}{15}
\newcommand{\anSettingnewIII}{12}
\newcommand{\anSettingnewIV}{4}
\newcommand{\anSettingnewV}{15}
\newcommand{\anSettingnewVI}{10}
\newcommand{\anSettingnewVII}{16}
\newcommand{\anSettingnewVIII}{3}

\newcommand{\anSettingrecAll}{37}
\newcommand{\anSettingrecI}{2}
\newcommand{\anSettingrecII}{5}
\newcommand{\anSettingrecIII}{1}
\newcommand{\anSettingrecIV}{4}
\newcommand{\anSettingrecV}{2}
\newcommand{\anSettingrecVI}{9}
\newcommand{\anSettingrecVII}{8}
\newcommand{\anSettingrecVIII}{6}



\title{%
  Processing of visual and non-visual naturalistic spatial information in
the "parahippocampal place area" \\
[1ex] \large Supplementary Information}

\author{
    Christian~O.~Häusler\textsuperscript{1,2{*}},
    Simon B. Eickhoff\textsuperscript{1,2},
    Michael Hanke\textsuperscript{1,2}}
% https://www.nature.com/sdata/publish/for-authors#other-formats

\maketitle
\thispagestyle{fancy}

\noindent
1. Psychoinformatics Lab, Institute of Neuroscience and Medicine, Brain \&
Behaviour (INM-7), Research Centre Jülich, Jülich, Germany\\
\noindent
2. Institute of Systems Neuroscience, Medical Faculty, Heinrich Heine University,
Düsseldorf, Germany\\
{*}corresponding author: Christian Olaf Häusler (der.haeusler@gmx.net)

\tableofcontents

\listoffigures

\pagebreak[4]

\section{Surface plots of individual results}

% individiaul results; intro
The results of the fixed-effects second-level analysis of the movie's primary
contrast (\texttt{vse\_new > vpe\_old}) and audio-description's primary contrast
(\texttt{geo, groom} > non-spatial noun categories) mapped onto subject-specific
cortical surfaces can be seen in Figure~\ref{fig:subjs_ao_av_c1_surf}.
% reconstruction
% \href{https://github.com/psychoinformatics-de/studyforrest-data-freesurfer.git}{dataset}
\href{https://surfer.nmr.mgh.harvard.edu}{FreeSurfer} v5.3.0
\citep{dale1999cortical} was used to reconstruct surfaces from T1-weighted
images and an additional high-resolution T2-weighted image
\citep{hanke2016freesurferdata}.
% projection
Thresholded $Z$-maps ($Z$>3.4; $p$<.05, cluster-corrected) were projected onto
the cortical surfaces using
\href{https://surfer.nmr.mgh.harvard.edu}{FreeSurfer's} \citep{dale1999cortical}
'mri\_vol2surf' command (unthresholded $Z$-maps are provided at \href{https://neurovault.org/collections/KADGMGVZ}{\url{neurovault.org/collections/KADGMGVZ}}).
% smoothing
After projection, individual PPA ROIs were spatially smoothed by applying a
Gaussian kernel with full width at half maximum (FWHM) of \unit[2.0]{mm}.

\begin{figure*}[tbp]
\centering
    \includegraphics[width=\linewidth]{figures/subjs_ao_av_c1_surf}
    \caption[Participant-specific surface plots of primary contrasts]
    {Fixed-effects individual-level GLM results ($Z$>3.4; $p$<.05
        cluster-corrected) projected onto reconstructed subject-specific brain
        surfaces.
        The results of the audio-description's primary
        $t$-contrast (blue) that compares geometry related nouns to non-       geometry related nouns spoken by the narrator
        (\texttt{geo, groom > all non-geo}) are overlaid over the movie's
        primary $t$-contrast (red) that compares cuts to a setting depicted for
        the first time with cuts within a recurring setting
        (\texttt{vse\_new > vpe\_old}).
        Black: outline of participant-specific PPA ROIs reported by
        \citet{sengupta2016extension} that was spatially smoothed by applying a
        Gaussian kernel with full width at half maximum (FWHM) of \unit[2.0]{mm}.
        }
    \label{fig:subjs_ao_av_c1_surf}
\end{figure*}


\section{Robustness of main findings across alternative contrasts}

In order to test the robustness of our approach, we created overall five
$t$-contrasts for the movie stimulus, and overall eight $t$-contrast for the
audio-description (see Table~5).
% \ref{tab:contrasts}).
% how
Contrasts differ in contrasted categories based on ``how well'' averaged events
within categories were considered to represent spatial and non-spatial
information, and the number of events in the stimulus.
% NeuroVault
Unthresholded $Z$-maps of all contrasts on a group level co-registered to the
group-template (MNI152 space) can be found at
\href{https://neurovault.org/collections/KADGMGVZ/}{\url{neurovault.org/collections/KADGMGVZ}}.

% AV PPA contrasts
% 1 PPA (l/r); RSC (l/r), LOC (l/r), VisC (l/r),
% intracalcarine, cuneal, ling.c
% 2 PPA (l/r); RSC (-/-), LOC (l/r), VisC (-/-)
% 3 PPA (l/r); RSC (-/-), LOC (-/-), VisC (-/r), occ. fusisf. g.
% 4 PPA (l/r); RSC (l/r), LOC (l/r), VisC (-/-), lingual g., occ. fusif. g.
% 5 PPA (l/r); RSC (l/r), LOC (l/r), VisC (-/-), occ. fusif. g.

% AV results: PPA; biggest overlap in posterior part of PPA group overlap  =
% temporal occipital fusiform cortex
Reliably across all five movie contrasts, we find bilaterally significant
clusters that overlap with the PPA group overlap (see
Figure~\ref{fig:stability-slices-volsurf}).
% AV: RSC & LOC
Several contrasts of the movie also yielded bilaterally significant clusters in
the ventral precuneal cortex and posterior cingulate gyrus (i.e. retrosplenial
complex; contrasts 1, 4, 5), and lateral occipital cortex (contrasts 1, 2, 4,
5).

% AD PPA contrasts
% 1 PPA (l/r); RSC (l/r), LOC (l/r)
% 2 PPA (l/r); RSC (l/r), LOC (l/r), r. sup.temp.C., r.fr.pole; putamen; etc
% 3 PPA (l/r); RSC (l/r), LOC (l/r),l&r sup.temp.C., anterior cing., paracing.
% 4 PPA (l/r); RSC (l/r), LOC (l/r)
% 5 PPA (l/r); RSC (l/r), LOC (-/-)
% 6 PPA (l/r); RSC (l/r), LOC (l/-), l&r sup.temp.C.; medial pref. c., precun
% 7 PPA (-/-); RSC (l/r), LOC (l/-), l&r sup.temp.C.; left frontal pole
% 8 PPA (-/-); RSC (-/-), LOC (-/-)

% AD: results
In results from the analysis of the audio-description, all contrasts except
contrast 7 and 8  yielded significant bilateral clusters in anterior regions of
the group PPA overlap (see Figure~\ref{fig:stability-slices-volsurf}).
% AD: RSC & LOC
Contrasts of the audio-description also yielded bilaterally significant clusters
in the ventral precuneus (all contrasts except contrasts 8) and lateral
occipital cortex (bilateral in contrasts 1 to 4; unilateral in contrasts 6 and
7).
% concluding stament regarding PPA, RSC, and LOC
Results across contrasts of both naturalistic stimuli indicate that our findings
regarding the PPA but also RSC and LOC are robust and do not depend on the
design of one specific contrasts.
% but:
Nevertheless, results are sensitive to the contrasted categories and the amount
of available data.
% example
For example, contrast 7 and 8 (\texttt{se\_new}, \texttt{se\_old} > non-spatial
categories, and \texttt{se\_new} > non-spatial categories) that used the most
heterogeneous categories (nouns indicating switches to other settings) and a low
amount of events yielded neither a significant cluster in the right-hemispheric
nor left-hemispheric PPA.
% concluding statement
Hence, investigators that use model-driven analyses have to consider how many
events a naturalistic stimulus may provide and how homogeneous the events to be
averaged might be.


\begin{figure*}[tbp]
\centering
    \includegraphics[width=\linewidth]{figures/stability-slices-volsurf}
    \caption[Group results across contrasts of movie and audio-description]
    {Overlap of significant clusters ($Z$>3.4; $p$<.05, cluster
    corrected) across all contrasts for both naturalistic stimuli.
    The audio-description's contrasts 1-8 (blue)
    are overlaid over the audio-visual movie's contrasts 1-5 (red;
    see Table~5)
    . %\ref{tab:contrasts}).
    % volume
    a) results as brain slices on top of the MNI152 T1-weighted head template,
    with the acquisition field-of-view for the audio-description study
    highlighted.
    % PPA
    For comparison depicted as a black outline, the union of the
    individual PPA localizations reported by \citet{sengupta2016extension}
    that was spatially smoothed by applying a Gaussian kernel with full
    width at half maximum (FWHM) of \unit[2.0]{mm}.
    % surface map
    b) results projected onto the reconstructed
    surface of the MNI152 T1-weighted brain template.
    After projection, the union of individual PPA localizations was
    spatially smoothed by a Gaussian kernel with FWHM of \unit[2.0]{mm}
    }
    \label{fig:stability-slices-volsurf}
\end{figure*}


\section{Incidental results}

Several contrasts yielded significant clusters in superior temporal cortices
(bilateral in contrasts 3, 6, 7; unilateral right in contrast 2).
% AD: auditory cortices
Further, contrasts that compare nouns indicating a switch to another setting
(categories \texttt{se\_new} or \texttt{se\_old}) with nouns from non-spatial
categories yielded significant clusters in primary and secondary auditory
cortices (bilateral: contrasts 3, 6, 7; unilateral right: contrast 2).
% interpretation 1
This suggests the presence of a low-level auditory processing confound that was
not completely captured by the employed nuisance regressors.
% possible explanation
The bias of the GLM contrasts towards low-level auditory perceptual processes
might be attributed to changes in the soundscape or the start of a new song that
accompany the narrator when he is indicating a switch to another setting.

%
Lastly, contrasts of the audio-description stimulus yielded clusters in regions
that were classified as statistically significant in two or less contrasts.
% contrast 2
In detail, contrast 2 yielded clusters in the anterior cingulate gyrus
(bilateral), right frontal pole, right frontal medial cortex, and right putamen.
% contrast 3
Contrast 3 shows clusters in the anterior cingulate gyrus (bilateral), left
frontal pole, and left insular.
% contrast 6
Contrast 6 shows one additional cluster in the right medial frontal cortex.
% contrast 7
Contrast 7 shows clusters in the left frontal pole and left frontal operculum,
left insular.
% concluding statement
Here again, significantly increased activation beyond areas known to be involved
in processing of spatial information could be attributed to stimulus features of
the naturalistic stimulus that were not sufficiently controlled leading to a
bias in the contrasts.


\section{Control contrasts}

% ---------- AV control contrasts ----------
% 6: nothing; 7 nothing; 8: nothing; 9: nothing; 10 se_new > se_old (nouns but
% in AV stimulus): LOC (l/r) right PPA, bilateral, sup. lat. occ. c., sup.
% parietal lobe (right)

% ---------- AD control contrasts ----------
% 9 PPA (-/-); RSC (-/r), LOC (-/-), left pars triangularis
%10 PPA (-/-); RSC (-/-), LOC (-/-)
%11 PPA (-/-); RSC (-/-), LOC (-/-),
%12 PPA (-/-); RSC (-/-), LOC (-/-), right posterior hippocampus
%13 PPA (-/-); RSC (-/-), LOC (-/-)

% intro
For each naturalistic stimulus, we created several control contrasts.
% movie contrasts: not cut condition vs. cuts
To test for the specificity of the main findings within the movie fMRI data, we
created four contrasts that compared hemodynamic activity during frames within
movie shots (\texttt{no\_cut} category) with movie cut related categories (see
Table~5).
% \ref{tab:contrasts}).
% results
None of these four contrasts yielded a significant cluster, suggesting that the
observed differential hemodynamic response in the PPA for events defined by
particular movie cuts is not observed for other arbitrarily selected time points
in the movie.

% \texttt{se\_new} (nouns) is r$\approx$0.3 correlated with \texttt{vse\_new}
% (cuts); \texttt{se\_old} (nouns) is r$\approx$0.4 correlated with
% \texttt{vse\_old}.
% \ref{fig:reg-corr})
The correlation of regressors across stimuli (Figure~4) showed
a moderate correlation among cuts to a setting depicted for the first time
(movie \texttt{vse\_new}) and nouns indicating a switch to a setting occurring
for the first time (audio-description \texttt{se\_new}), and cuts to a
recurringly depicted setting (movie {\texttt{vse\_old}) and nouns indicating a
switch a recurring setting (audio-description \texttt{se\_old}).
%
Hence, we computed an indentical contrast ({\texttt{se\_new} >
\texttt{se\_old}), comparing the audio-description's voice-over narrator nouns
for the movie data and the audio-description.
% results
While for the movie data the contrast yielded a significant cluster in the right
PPA, not in its anterior portion, and bilateral clusters in the superior lateral
occipital cortices (the right hemispheric cluster extending into the superior
parietal cortex), no significant differences where found for the
audio-description (unthresholded contrast maps are available on NeuroVault).
%
These results suggest that in the absence of the actual visual stimulation, the
narrator's description of scene properties correlated with setting changes did
not co-occur with increased hemodynamic activity in areas associated with the
processing of spatial information.

% AD stimulus
Finally for the analysis of the audio-description, we computed control contrasts
based on events of the annotated movie cuts (see Table~5) in order
to evaluate the influence of stimulus features correlated with these visual
events that remain present in the audio-description, such as changes in the
soundscape.
% results
Two of five contrasts yielded significant clusters.
% contrast 9
Contrast 9 (\texttt{vse\_new > pe\_old}) revealed one significant cluster in the
left inferior prefrontal cortex (pars triangularis) and right ventral
precuneus/posterior cingulate gyrus.
% contrast 12
Contrast 12 (\texttt{vse\_new} > \texttt{vse\_old}, \texttt{vpe\_old}) revealed
one cluster in the right posterior hippocampus.
%
The absence of observable hemodynamic response differences in the PPA suggests
that without the visual information in the movie stimulus, no, or substantially
less, processing of stimulus content with respect to spatial information
co-occurred with the timing of the movie cuts.


{\small\bibliographystyle{unsrtnat}
\bibliography{references}}

%\begin{thebibliography}{1}
%\expandafter\ifx\csname url\endcsname\relax
%  \def\url#1{\texttt{#1}}\fi
%\expandafter\ifx\csname urlprefix\endcsname\relax\def\urlprefix{URL }\fi
%\providecommand{\bibinfo}[2]{#2}
%\providecommand{\eprint}[2][]{\url{#2}}
%
%\bibitem{cite1}
%\bibinfo{author}{Califano, A.}, \bibinfo{author}{Butte, A.~J.},
%  \bibinfo{author}{Friend, S.}, \bibinfo{author}{Ideker, T.} \&
%  \bibinfo{author}{Schadt, E.}
%\newblock \bibinfo{title}{{Leveraging models of cell regulation and GWAS data
%  in integrative network-based association studies}}.
%\newblock \emph{\bibinfo{journal}{Nature Genetics}}
%  \textbf{\bibinfo{volume}{44}}, \bibinfo{pages}{841--847}
%  (\bibinfo{year}{2012}).
%
%\bibitem{cite2}
%\bibinfo{author}{Wang, R.} \emph{et~al.}
%\newblock \bibinfo{title}{{PRIDE Inspector: a tool to visualize and validate MS
%  proteomics data.}}
%\newblock \emph{\bibinfo{journal}{Nature Biotechnology}}
%  \textbf{\bibinfo{volume}{30}}, \bibinfo{pages}{135--137}
%  (\bibinfo{year}{2012}).
%\end{thebibliography}

\end{document}
